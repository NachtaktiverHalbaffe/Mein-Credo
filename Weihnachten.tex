% !TEX root = Lehren__von_Jesus_Christus.tex
\chapter{Weihnachten}
Bevor das Leben Jesu betrachtet werden kann muss erstmal dessen Geburt bzw.\ dessen kommen betrachtet werden. Zunächst geht es also um Weihnachten und die Zeit davor: Advent. Ziel ist es, herauszuarbeiten, wie Advent und Weihnachten eigentlich gefeiert werden sollte, losgelöst von Traditionen, Erwartungen und Kommerz. 

\section{Advent}\label{sec:Advent}
Verwendete Bibelstellen:
\begin{itemize}
\item Römer 13, 11--14
\item Matthäus 11, 2--6
\item Matthäus 24, 29--31
\item Lukas 21, 25--28
\item Philipper 4, 2--9
\item Offenbarung 22, 6--21
\end{itemize}
Die Geschichte hinter dem Advent ist weit bekannt und wird im Abschnitt\ \ref{sec:Weihnachten}: Weihnachten genauer durchleuchtet. Im Folgenden wird daher nur das nötigste Kontextualisiert und Vorhergegriffen. Die Bibel bietet keine breite Grundlage für den Advent und der Weg vor Weihnachten wird zudem im Lukas- und Matthäusevangelium konträr dargestellt. Desweiteren ist die Reise von Nazareth nach Bethlehem wahrscheinlich frei erfunden. Jedoch ist eine Zeit des Vorbereitens und Wartens auf das eigentliche Ereignis nicht falsch, da in Anbetracht des Weihnachtstresses ein harter Einstieg in Weihnachten viele Gläubige das Fest nicht in seiner vollen Form erleben lässt, da sie direkt hineingeworfen werden. Die Vorbereitungszeit ist daher im Kerngedanken eine gute Idee, auch wenn es nicht direkt auf der Bibel fußt.
\\~\\
Bei der klassischen Adventsdiskussion geht es immer um Unterwegs sein (wenn auch nicht im Sinne von Reisen). Alles ist vor und während der Geburt unterwegs, dies dürfte jeden klar sein, die selbst schon bei der Zeit einer Schwangerschaft dabei waren. Wo ein Weg ist, ist auch ein Ziel. Damals war das Ziel die Geburt des Messias. Doch wenn Advent ein Weg ist, was ist dann heute das Ziel, wenn das damalig Ziel bereits erreicht wurde? Wie sieht der Weg aus? Dies sind die wesentlichen Fragen der Betrachtung.


\subsection{Warten auf den Erlöser}
Wichtig sind die Hintergründe der dortigen Zeit. Das Königreich Israel hatte seinen Zenit schon längst überschritten und ist längst zerfallen. Es wurde von dem Römern besetzt und war eine Provinz, die Steuer entledigen durfte und bedingungslosen Gehorsam leisten musste. Die Folge war große Armut und viel Leid. Eine dunkle Zeit in Israel, mit der sich nicht jeder abfinden wollte. Es gab bereits kleine Rebellengruppen, die sich auflehnten und alle erwarteten den Messias, der sie von dieser Situation (oder besser gesagt: Gefangenschaft) befreien würde. Israel wartete auf seinen Erlöser. Sie waren Gefangen in einen Gefängnis. Ein Gefängnis der Besatzung, gegen des sich Israel zunächst nicht wehren konnte.  
\\~\\
Sie waren gefangen, genauso wie Johannes der Täufer. Er wartet auf den Erlöser, der ihn und sein Volk befreien würde. Genauso wie Johannes und Israel zu ihrer Zeit sind auch wir in unseren eigenen kleinen Gefängnis gefangen. Vielleicht ganz ähnlich wie Johannes an einen Ort, an den wir eigentlich nicht sein wollen. Ein Gefängnis aus Streit zwischen zwei Glaubensgeschwistern wie zwischen Synteche und Evodia, die ursprünglich zusammen für die Botschaft Gottes gekämpft haben. Ein Gefängnis, was wir uns selber erbaut haben durch z.B. ein gestörtes Selbstbild. Genau diese Gefängnisse, aus denen wir innerlich am lautesten schreien und am liebsten ausbrechen würden. Oder wie es Paulus im Römerbrief formuliert: Wir schlafen in einer Finsternis und warten darauf, aufzuwachen. Hierbei sind Mächte oder Lebensweisen gemeint, die schlecht für uns sind und uns von Gott und seinen Vorstellungen abbringen oder mit tiefst dogmatisch befleckten Worten: Sünde.
\\~\\
Aus diesen Schlaf sollen wir erwachen. Laut Paulus sei die Zeit gekommen, daraus aufzuwachen, denn die Rettung ist nah. Wir sollen Jesus als den Erlöser in unseren Leben aufnehmen und nicht um unser leibliches Wohlergehen sorgen. Wir sollen keine unnötigen Feste wie z.B. das Oktoberfest feiern, Alkohol nicht missbrauchen, keine sexuelle Verwerflichkeiten begehen und Streit \& Eifersucht meiden. Wir sollen dieselbe Gesinnung gegenüber den Herrn haben und nicht darüber streiten, wie es Synteche und Evodia machten (Heutzutage: Im ökumenischen Sinne). All dies aus den einzigen Grund: Jesus als unser Erlöser ist nahe und ist gekommen, um uns zu erretten. Johannes Warten im Gefängnis auf seinen Erlöser hat sich also gelohnt. Denn Jesus als der Erlöser ist am kommen.
\\~\\
Heutzutage warten wir auf eine andere Weise auf den Erlöser, dennoch ähnelt es sich. Der Messias ist bereits gekommen und brachte, bringt und wird fortan Erlösung bringen. Dennoch warten wir tagtäglich auf einen Erlöser, der uns aus unseren Schlaf aufweckt und uns aus dem Gefängnis befreit. Wir warten genauso wie Johannes der Täufer oder die Israeliten auf den Erlöser, auch wenn sich die Probleme und die Gefängnisse, in denen wir gefangen sind, geändert haben. Jedoch ist Jesus bereits in unser Leben eingetreten und es liegt an uns, ihn als unseren Erlöser anzunehmen und Erlösung zu finden. Wie es Paulus passend formuliert ``Nein, zieht Jesus Christus, den Herrn, an wie ein neues Kleidungsstück'' (Römer 14, 11--14; Basisbibel).

\subsection{Feiern des Advents im heutigen Kontext}
So gerne man auch den Advent feiern möchte, so sehr ist das Fest auch nicht biblisch. Die Bibel gibt keinen Ablauf und kein Leitbild, an dem man festhalten könnte. Der Advent ist wie so ziemlich alle christlichen Feste ein Menschenwerk, um sich an etwas zu ermahnen, zu erinnern und/oder daran sich zu erfreuen. Da das Fest im Mittelalter und somit in einen anderen Kontext eingeführt wurde, lohnt es sich unter Berücksichtigung unseres heutigen Kontextes diese Zeit des Erwartens in seinen Ritualen, seinen Grundgedanken und in seinen Leitbild neu zu überdenken.  
\\
Im Kern ist der Advent ein Fest, um das kommen des Erlösers vorzubereiten. Oder in anderen Worten: Wir bereiten uns auf Weihnachten vor und auf Jesu wiederkommen in der sogenannten Endzeit. Es ist der Gedanke, dass Jesus kommen wird, um uns zu erretten, ähnlich wie damals die Menschen vor der Geburt Jesu. Der Kerngedanke ist also, auf was bevorstehendes zu warten und uns darauf vorzubereiten. Vorbereiten auf das Weihnachtsfest zur Erinnerung an Jesu Geburt. Vorbereiten auf die Endzeit, in der unsere Welt und Gottes Reich eins werden. Doch wie bereitet man es vor? Wie stellt man sich darauf ein? 

\subsubsection*{Vorbereitung auf Jesus Kommen}
Es klingt erstmal perplex, dass Jesus (oder auch: Der Erlöser) kommen soll. Schließlich ist dies alles schon eingetreten. Dennoch sinnen wir tagtäglich auf Erlösung. Die Große Erlösung, auf die wir uns besinnen können, ist die Endzeit. Wir haben das Versprechen, dass wir zu Gott kommen und Teil seines Reiches werden. Alles Leid, alle Plagen, alle Ungerechtigkeiten und all das, was uns auf Erlösung sinnen lässt, wird sich dort in Luft auflösen. All dies wird keine Rolle mehr spielen. Wir wissen nicht, wann dieser Tag kommen wird. Wir wissen nur, dass er kommen wird und dass wir ihn erkennen werden und dass wir zweifelsfrei dort Jesus erkennen werden, der kommen wird, um uns zu retten.
\\~\\
Doch das Versprechen in die Endzeit ist ein explizites Versprechen für ein Event in der Zukunft. Ein Event, wo wir -genauso wie Israel damals auf seinen Messias gewartet hat- auf das Kommen Jesu warten, ohne zu wissen wann und ob es zu unseren Lebzeiten auf Erden geschehen wird. Genau wie damals werden Leute die Endzeit zu früh sehen, ihr komplette Hoffnung mehr auf das Event als auf Gott selber setzen oder radikalisieren sich an den Gedanken eines frühzeitigen Endes. Im Endeffekt können wir bezogen auf die Endzeit nicht mehr machen als damals: auf Gott vertrauen und demütig warten.
\\ 
Doch diese Botschaft hat auch für die Gegenwart eine elementare Bedeutung. Auch jetzt können wir auf Gottes Versprechen bauen, dass auf schlechte Zeiten gute Zeiten folgen werden. Dass er bei Krisen bei uns ist und uns trägt. Und dass wir, wenn wir vermeintlich am Ende sind, Jesus sehen werden und er kommen wird, um uns zu retten. 
\\~\\
In gewisser Weise kann man auch Weihnachten mit einer Art Endzeit vergleichen. An keinen anderen Feiertag haben mehr Menschen frei als an diesen. Um 18 Uhr am heilig Abend steht das öffentliche Leben still und die Leere der Straßen gleichen einer Apokalypse. Man rückt mehr zusammen, Streit und Feindschaften werden beigelegt (jedenfalls die, die man temporär ablegen kann). Jeder füllt seine Kühlschränke auf und rüstet sich für die Feiertage. Weihnachten kann als ein jährliche Endzeit betrachtet werden, auch wenn sie nicht mit den Ausmaßen der tatsächlichen Endzeit vergleichbar ist. Es können also in einen begrenzten Rahmen Parallelen zur Schilderungen der Endzeit gezogen werden.

\subsubsection*{Vorbereitung auf das Fest}
In unseren modernen Gesellschaft ist es Tradition geworden, sich auf ein Weihnachtsfest des unbedachten Massenkonsums und des unbedachten Liebesschenken vorzubereiten. Die Läden müssen möglichst lang offen haben und möglichst viel Umsatz machen. Jedermann soll seinen Mitmenschen was schenken, um ihn so wertzuschätzen. Eine Idee des Marktes. Das Coca Cola den Weihnachtsmann und somit den Brauch des Schenkens als Liebesbeweis erfunden hat, um damit ein Fest zu schaffen, dass ein Rekordumsatz mit sich bringt, ist wohlbekannt. Nicht die Liebe steckt im unbedachten Massenschenken dahinter, sondern Materialismus, Egoismus und das Bestreben nach Reichtum. Irdische Werte und Bedürfnisse, die vergänglich sind. Sicherlich darf man seinen liebgewonnenen Menschen eine Freude machen, indem man ihnen etwas schenkt. Doch reichen dafür wenig gut ausgewählte und mit Liebe gemachte Geschenke. Die Suche nach diesen Bedarf natürlich einen Zeit- und Arbeitsaufwand, doch darf dieser nicht die Adventszeit noch dein Leben zu sehr bestimmen. Hinter dem Weihnachtsfest steckt eine anderer Gedanke und dieser Bedarf keiner großer Vorbereitung. Man muss lediglich den anderen das geben, wovon man selbst zu viel hat.
\\
Für viele heißt es, dass man die Menschen um sich scharen soll, die man liebgewonnen hat, um mit ihnen ein schönes Fest der Liebe zu feiern und nicht das Fest der Geschenke. Dies Bedarf lediglich einen organisatorischen Aufwand und einen geringen Arbeitsaufwand am Fest selber (je nachdem, wie ausgeschmückt man es feiert). Das Fest der Liebe, die aus der Gnade Jesus Christi resultiert, bedarf keinen aufwändig gemachten und teuren Essens. Es bedarf keiner teuren Geschenke. Es bedarf nur der liebevollen Gemeinschaft und für dieses bedarf es nur ein einziges und heutzutage wertvollste Geschenk: Zeit. Nicht die Zeit, in der man 100 Personen mit ihren Festen aus einen Zwang abklappert und jeder Person nur eine Stunde gibt. Es bedarf die Zeit, die man freiwillig in den Maße investiert, wie jeder sie braucht. 
\\~\\
\textbf{Freiwillig}, \textbf{wahre Liebe} und \textbf{Gemeinsam}: Das sind die drei Worte, die das Fest beschreiben.\ \textbf{\textit{Freiwillig}}, weil es nur dann ein Fest der wahren Liebe sein kann, wenn jeder aus freien Stücken da ist und deswegen seine Zeit (und sein Geld) darin investiert. Eine solche wahre Liebe kann nur unter Freiheit bestehen.\ \textbf{\textit{Wahre Liebe}}, weil wir uns an diesen Fest auf Jesus und seine dargebrachte Gnade besinnen, durch die wir Gott in seiner Dreieinigkeit und unsere Mitmenschen erst richtig Lieben können. Und schlussendlich \textbf{\textit{Gemeinsam}}, weil man seine Mitmenschen und Gott nur dann Lieben kann, wenn man seine Zeit gemeinsam mit ihnen verbringen kann. Alles andere kann keine wahre Liebe sein.
\\~\\
Damals bei der Geburt Jesu scharten sich auch viele Menschen um das Kind. Niemand fühlte sich gezwungen, Jesus etwas zu schenken, auch wenn sie vielleicht selber nichts hatten. Sie gaben nur das, was sie in Überfluss hatten. Bei den heiligen drei Königen dürften dies auch wertvolle Sachgaben gewesen sein, da sie wohlhabend waren. Die Hirten dagegen brachten nur ihre Hoffnung. Man soll nur das geben, was man selbst im Überfluss hat oder wo man selbst nicht Not leidet und bei den meisten dürfte das bei den Feiertagen ihre Liebe, Zeit und Gemeinschaft sein. 
\\~\\
\textit{Wenn der gute Wille da ist, ist jeder willkommen mit dem, was er hat und man fragt nicht nach dem, was er nicht hat. Denn es geht nicht darum, dass ihr in Not geratet, indem ihr anderen helft, es geht um einen Ausgleich. Im Augenblick soll euer Überfluss ihren Mangel abhelfen, damit auch ihr Überfluss einmal euren Mangel abhilft.\\ 1. Korinther 8, 12 ff.}

\subsubsection*{Aus eigene Gefängnissen ausbrechen}
Neben den Gedanke des Vorbereitens auf das Kommen Jesu ist wie bereits ausgearbeitet auch ein Kerngedanke von Weihnachten das Warten auf Erlösung. Das Ausbrechen aus den eigenen Gefängnis mithilfe von unsern allmächtigen und barmherzigen Vater in Form von Jesu Christus. Wie so ziemlich alle Lehren von Jesus Christus sind diese auf den Alltag und nicht nur auf ein Fest bezogen und sollten tagtäglich gelebt werden. Jedoch kann es nicht schaden , v.a im Anbetracht der Trägheit des Menschen, sich eine Zeit mit einer Sache genauer und intensiver zu beschäftigen und so jährlich aus den Trott auszubrechen, in den man sich schnell verfängt. Und welches Fest außer Advent eignet sich besser dafür, uns mit unseren eigenen Gefängnissen auseinanderzusetzen und auf den Ausbruch daraus mithilfe von Jesus Christus, der auf die Erde kommen wird, zu hoffen? Auf einen Tag in der Zukunft zu hoffen, in der unser Gefängnis nicht mehr ist?
\\~\\
Für mich persönlich ,um mal aus der Ebene der allgemeinen Lehre auszutreten und es auf mein eigenes Leben zu beziehen, heißt das, zu Advent aus einen eigenen Gefängnis auszubrechen. Sich mit einen tiefer liegenden Problem in seinen eigenen Leben zu beschäftigen und es anzupacken. Für Heilung zu beten und mit unseren Vater darüber im tiefen Gespräch sein. Im Gegensatz zum Fasten geht es hier nicht nur darum, Sachen die einen nicht gut zu tun aus seinen Leben zu entfernen oder auf etwas verzichten. Im Gegenteil, es geht auch um Sachen, die wir bereits machen und nicht schlecht sind, aber die für uns persönlich zu kurz kommen, zu intensivieren. Man kann z.B. Sport machen, was nichts schlechtes ist, aber mit dessen Regelmäßigkeit man (noch) nicht zufrieden ist. Auch ist im Gegensatz zum Fasten hier das Ziel eine nachhaltige Veränderung.
\\~\\
Dabei dient folgende Struktur -angelehnt an das Kommen Jesu- als Vorstellung:
\begin{enumerate}
\item \textbf{Das kommen wir angekündigt}: Zum Anfang des Advent sollte klar sein, welchen Gefängnis man sich widmen will. Es braucht keine Struktur, kein Plan, keine Methodik. Es bedarf lediglich einen Ziel
\item \textbf{Das kommen Jesu wird mit der Schwangerschaft von Maria konkret}: In der ersten Adventswoche soll mit tiefen Fokus auf Gebet und Stille Zeit ein Plan gefasst werden. Eine Analyse des Problems und wo man aktiv werden muss oder Gott uns sagt, dass wir aktiv werden sollten
\item \textbf{Auf den Weg gehen nach Bethlehem}: In der zweiten und dritten Adventswoche soll der Weg folgen. Das heißt ganz konkret den Plan umsetzen, den man sich in der ersten Adventswoche zusammen mit Gott erarbeitet hat.
\item \textbf{Ankommen in Bethlehem und erkennen, dass Jesus die Welt dauerhaft ändern wird}: Die letzte Woche soll dazu dienen, einerseits langsam sich Richtung Weihnachten zu orientieren, andererseits bewusst zu machen, dass die ersten drei Adventswochen wie bei der Reise von Maria und Josef nur der Anfang der Geschichte waren. Jetzt sollte man sich Gedanken machen, wie man aus den Gefängnis draußen bleibt und nicht zurück wandert. Konkret heißt dies, Routinen zu entwerfen, die einen dauerhaft helfen und Kontrollinstanzen zu finden, in denen man reflektieren kann, ob man sich noch auf den richtigen Weg befindet.
\item \textbf{Jesus rettet die Welt für immer}: Irgendwann wird sich (hoffentlich) der Zeitpunkt einstellen, ab den man weiß, dass man es dauerhaft geschafft hat, aus seinen Gefängnis auszubrechen. Jetzt kann man seine Routinen, sofern sie nicht mehr erforderlich sind, ablegen und die Kontrollinstanzen, die man sich selbst geschafft hat, abschaffen.
\end{enumerate}
Diese Struktur ist für mich persönlich als Leitfaden entstanden. Es kann von jeden adaptiert werden, sollen aber keinen Anspruch für die Allgemeinheit haben. Außerdem soll es nur eine grobe Struktur geben, welche sich je nach Problemstellung oder Lebenslage anpassen kann z.B. wie lange eine Phase geht.

\subsection{Fazit}
Advent ist eine Zeit des Unterwegs sein. Das Ziel ist das Weihnachtsfest, die Wiederkunft Jesu in der sogenannten Endzeit und die Erlösung bzw.\ das Ausbrechen aus den eigenen Gefängnissen. Der Weg zeichnet sich (in Reihenfolge der Aufzählung der Ziele) -in meinen persönlichen Fall- durch das Ausbrechen aus einem eigenen Gefängnisses aus. Ein weiterer Weg ist die Vorbereitung von Weihnachten selber, wobei man darauf achten sollte, nur das zu schenken, wovon man selbst im Überfluss hat und das Fest so zu gestalten, sodass kein Zwang entsteht, sodass wahre anstatt vorgegaukelter Liebe in Vordergrund steht und es ein Fest der Gemeinschaft ist. Der letzte Weg zur Endzeit zeichnet sich nicht besonders aus. Zwar hat Weihnachten selber an manchen Stellen endzeitlichen Charakter, doch außerhalb einer inhaltlichen Auseinandersetzung gibt es keinen richtigen Weg, sondern nur ein Versprechen Gottes und eine dauerhafte Einstellung des Gläubigen. 

\newpage
\section{Weihnachten}\label{sec:Weihnachten}
Verwendete Bibelstellen:
\begin{itemize}
\item Lukas 1--2
\item Matthäus 1--2
\item Johannes 1, 1--18
\end{itemize}

\subsection{Kontextualisierung}
Bereits beim Abschnitt\ \ref{sec:Advent}: Advent wurde ein wenig auf den damaligen Kontext um die Zeit von Jesu Geburt eingegangen. Jetzt soll der Kontext genauer betrachtet werden. Als erstes ist die römische Besatzung zu betrachten. Die politischen und wirtschaftlichen Strukturen des römischen Reiches sind (im überschaubaren) Sinne ähnlich zu den heutigen. Das römische Reich kann als ein antikes Modell einer globalisierten Welt verstanden werden. Das politische Geschehen wurde in den wichtigen Entscheidungen zentral in Rom von reichen und/oder einflussreichen Politikern bestimmt und tagespolitische Entscheidungen bzw. Verwaltungen dezentral von eingesetzten Statthaltern bestimmt, die Rom gegenüber zu Loyalität verpflichtet waren, dafür aber auch einbezogen wurden in den Reichtum und Politik. Dies erklärt auch die Nähe von den Hohepriestern (und damit den Sadduzäern) als auch den politischen Machthabern im damaligen Israel zu Rom und das Vorgehen gegen Jesus (und andere Gegner Roms), wenn Rom öffentlich angefeindet wurde oder ein Aufstand angezettelt werden sollte.
\\
Das römische Reich expandierte durch militärische Feldzüge und dies sehr erfolgreich. In eroberten Gebieten wurde den dort ansässigen Machthabern und religiösen Führer Partizipation und Reichtum angeboten, solange sie Rom treu ergeben blieben und ``nach ihren Willen tanzten''. Auch wurde die dortige Infrastruktur gestärkt inform von Straßen, Viadukten o.ä., solange sie Rom Vorteile erbrachten. Die dadurch entstehenden hohen Kosten und die Kosten des Militärs mussten finanziert werden.
\\~\\
Das wirtschaftliche System beruhte auf militärische Expansion. Nach Eroberung wurden die Reichtümer (auch religiöser Natur) geplündert und an Rom weitergegeben. Außerdem gab es ein ausgeklügeltes Steuer- und Abgabensystem. Je näher ein Standort an den Machtzentren des römischen Reichs lagen, desto mehr wurde durch Sozialmaßnahmen versucht, die Bevölkerung bei Laune zu halten. Befand man sich aber außerhalb dieser Zentren, dann waren die Abgaben erbarmungslos und die Bevölkerung musste um ihr Überleben kämpfen. Auch gab es schon damals in Rom Geschäfte mit Spekulationen auf Silber o.ä., was Auswirkungen auf den Wert eines Gutes in eines der Provinzen haben konnte.    
\\~\\
Israel zählte zu den großen Verlieren im römischen Reich. Zusätzlich wurde das Land schon vor den Römern von z.B. den Griechen, die auch nach ihrer Herrschaft noch einen sogenannten hellenistischen Einfluss hinterließen, und anderen Nationen und deren Eroberungsfeldzüge geplagt und zugrunde gerichtet. Viele Juden mussten sogar fliehen und lebten in einer Diaspora. Für Israel hatten die Zustände um die Geburt und das Wirken Jesu endzeitliche Zustände. Dies erklärt auch die Weltansicht einer bösen Welt und einer nahestenden Endzeit und Wiederkunft des Messias, auch nach Jesu Tod hinaus in den Urgemeinden. Gerade diese Zustände riefen viele Aufstände, radikale Bewegungen und Reformbewegungen auf den Plan, einige angeführt von Gläubigen mit den Anspruch, der Messias zu sein. Jesus war also nicht der erste, der behauptete Gottes Sohn zu sein oder der den Glauben reformieren wollte.

\subsection*{Die Evangelien}
Das Evangelium nach Lukas versteht sich nicht als historisch im heutigen Sinne (sondern im damaligen Sinne), aber er versuchte, die Geschichte Jesu in die damaligen politischen und gesellschaftlichen Geschehen einzuordnen, auch wenn ihn dabei ein paar Fehler unterlaufen sind. Der Bericht nach Lukas kann also am ehesten in einen begrenzten Rahmen als ``historischer'' angesehen werden, hat aber keinen Anspruch, eine faktische Historie darzustellen.
\\~\\
Das Evangelium nach Matthäus stellt gewisse jüdische Gruppen negativ dar. Daher ist anzunehmen, dass er von einer jüdischen Gemeinschaft ausgeschlossen wurde. Daher wird in Teilen des Matthäusevangeliums der Konflikt zwischen den ersten Judenchristen und Juden deutlich. Der Fokus liegt darin, Jesus als den Messias darzustellen und daher bezieht er sich auch auf Prophetien des alten Testamentes und er grenzt sich von den ``alten'' Judentum ab. Daher ist kritisch zu betrachten, ob diese Prophetien wirklich geschehen sind oder ob Matthäus, für damalige Verhältnisse in legitimer Weise, einige Sachen zurechtgerückt hat, sodass sich die Prophetien erfüllen.
\\~\\
Es ist anzunehmen, dass Matthäus und Lukas jeweils das andere Werk nicht kannten, was die Unterschiede in der Kindheitsgeschichte Jesu erklärt. Aufgrund der damaligen geographischen Einschränkungen und unterschiedlichen Quellmaterial kann nicht aus den Bibeltext heraus geklärt werden, welche Überlieferung historisch korrekter ist, jedoch haben die Schriften sowieso keinen historischen Anspruch im heutigen Sinne und es muss entschieden werden, wie wichtig korrekte historische Fakten für den Sinn des Weihnachtsfestes sind.
\\~\\
Die Apokryphen hatten auch einen nennenswerten Einfluss auf das heutige Bild von Weihnachten. Der Ochse und der Esel wurden aus den Apokryphen überliefert. Aufgrund der teilweise brüchigen Überlieferung sowie Unstimmigkeiten bei der Inspirationsfrage als auch späte Verfassungsdaten von bis zu 600 n.Chr.\ werden diese nicht mit einbezogen in die Betrachtung.
\\~\\
Da sowohl Lukas als auch Matthäus keinen vordergründigen Anspruch auf korrekte Historizität haben, soll der Kindheitsbericht nicht als historisches Ereignis, sondern als theologische Grundlage für die Evangelien verstanden werden. Da aber Markus keine Kindheitsgeschichte enthält und dies das einzig gemeinsame Geschichtsgut von Lukas und Matthäus ist, aber beide nicht die Evangelien der anderen kannten, ist anzunehmen, dass es sich bei den Gemeinsamkeiten durchaus um Ereignisse handeln kann, die durchaus stattgefunden und sich mündlich bis zu den beiden getrennt herangetragen haben können. Jedoch sollten nicht alle Inhalte der Kindheitsgeschichten als wahre Ereignisse interpretiert werden. Um den Untergang Pompejis gab es auch Sagen und Legenden, aber die Stadt hat wirklich existiert und sie ist wissenschaftlich belegt durch einen Vulkanausbruch zerstört werden. Wenn also beide Geschichten übereinstimmen, kann durchaus eine gewissen Historizität des Ereignisses in Betracht gezogen werden, in Details und Unterschieden jedoch kann es sich um, für damals in den Religionen nicht unüblichen, theologischen Stilmittel handeln, die etwas betonen sollen.

\subsection{Engelserscheinungen von den Eltern von Jesus und Johannes den Täufer}
Engel sind göttliche Wesen, die über den Menschen stehen und Gottes Angesicht gesehen haben. Dieser göttliche Anspruch stärkt die theologische Botschaft, die Lukas und Matthäus zu vermitteln versuchen. Die Existenz von Engeln kann wissenschaftlich nicht bewiesen noch widerlegt werden. Die Existenz wird als widerlegt angenommen und die Engel als Metapher für die guten Wesenszüge von Gott angenommen (siehe \ref{sec:Teufel}). Jedoch kann man Aussagen zu deren Botschaft treffen. Da die Engel in einen klassischen alttestamentlichen Verkündigungsschema reden, ist anzunehmen, dass es sich bei den Engeln um theologische Stilmittel handelt und nicht um historische Ereignisse oder dass der Sinn in einen neuen rhetorischen Rahmen gepackt wurde. Wahrscheinlich ist letzteres Wahrscheinlicher, da sich letzten Endes die Urgemeinde, von der Maria ein Teil war, dies nicht angezweifelt hat und die Verkündigung auch erfüllt wurde. Jedoch sind aus meiner Sicht beide Ansichtsmöglichkeiten plausibel und es gibt nicht die eine Wahrheit. Die Engel bestätigen also den Auftrag sowohl von Johannes den Täufer als auch von Jesus sowie deren Rollen.
\\~\\
Lukas und Matthäus geben Jesus und Johannes Familie einen göttlichen Stammbaum, die  auf das Königsgeschlecht Davids und auf Aaron zurückgehen. Es ist unwichtig, ob diese Abstammung biographischer Art ist. Zu einen gab es damals ein anderes Verständnis von Genealogie (man betrachte den Kult um Heidenkönige als Gottessöhne heidnischer Götter, obwohl sie offensichtlich biologisch von irdischen Leben abstammen) und zu anderen ist hierbei die theologische Genealogie wichtig. Gott erwählte Israel als sein Volk, Moses als sein Prophet und Aaron als seinen Hohepriester. Diese salbten David zum König, der durch Gott bestätigt und bestraft wurde, als dieser sich von ihn abgewandt hatte. Hierauf folgt auf einer Zeit der Verdammung Jesus als der Nachfolger Davids als nächster großer König für die Ewigkeit und Johannes als der Nachfolger Elias soll sein Kommen vorbereiten. Dieser irdische (horizontale) Stammbaum wird hierbei bei Jesus durch Gott von oben (vertikale) durchschnitten. Die Zeit der irdischen Könige ist vorüber und der letzte König kommt für die Ewigkeit und ist kein anderer als Gottes Sohn selber. Der universale Anspruch wird bei Matthäus betont durch die Erwähnung des ``Buches der Abstammung'', das zuletzt bei der Schöpfungsgeschichte der gesamten Menschheit erwähnt wurde und somit zeigt, dass es sich um den König der Menschheit handle. Ebenso zeugt die Abstammung von Abraham davon, dass Jesus Anspruch weltweit gilt, da Abraham ebenso für alle Völker berufen wurde. Dieser Beweis durch eine Abstammung war Matthäus so wichtig, dass er den Stammbaum mit einer ``heiligen Mathematik'' versah, sodass nicht an diesen gezweifelt werden dürfe. Jedoch bedeutet dies nicht, dass eine Abstammung Jesu von David nicht existiert, sondern dass man es mittlerweile einfach nicht mehr sagen kann.
\\~\\
Der universale Geltungsanspruch wird auch durch die Wahl der Herkunft von Jesus unterstrichen. Jesus stammt aus Nazareth, ein Ort niedriger Bedeutung und der gemischten Völker. Jesus wird nicht im Tempel oder im Allerheiligsten verortet, die Orte die dem Allerheiligsten und nur Gott vorbehalten blieben, sondern einen Ort, wo jeder sein darf. Die einzige Verbindung zum Königtum bleibt also bei der Abstammung. Dies bestätigt das Alte und zeigt eindeutig, dass eine neue Zeit anbricht, wobei das Alte nicht aufgehoben wird, sondern eine Verheißung zusätzlich erfüllt wird. Auch zeigt dies, wie tief sich Gott begibt. Er stammt aus einen Dorf, das in Israel religiös als auch wirtschaftlich nahezu komplett unbedeutend war. Er stammt aus einen Dorf, wo jedermann als Letztes nach ihn gesucht hätte.

\subsection{Verheißung auf Erlösung}
Mit der Wahl der Eltern von Johannes liefert die Verheißung direkt Erlösung bzw.\ den Ausbruch aus einem Gefängnis (siehe Advent): Elisabeth  wird erlöst aus den Schicksal der Kinderlosigkeit. Dies war zur damaligen Zeit wirklich ein Gefängnis, da unfruchtbare Frauen als von Gott bestrafte galten und daher in der jüdischen Gesellschaft sehr diskriminiert wurden. Die Berichte erwähnen ihr Frommes leben. Dies zeigt: Die Eltern von Johannes litten zu Unrecht und hatten das ``Recht nach Erlösung'', die Gott ihnen zukommen ließ.
\\~\\
Eine weitere Erlösung wird durch Lukas Einordnung der Geburtsgeschichte in die Zeit der römischen Herrschaft durch Herodes angedeutet. Lukas stellt Jesus als den Erlöser der römischen Herrschaft entgegen. Jesus ist der Erlöser, der von den römischen Reich verfolgt und gekreuzigt wird und damit vermeintlich verliert, jedoch in der Auferstehung den endgültigen Sieg davon trägt.
\\ 
Auch bringt Gott den Eltern Erlösung, bevor sie wissen dass sie in eine  Gefängnis sind. Er erklärt Joseph die Geisteszeugung und verhindert so die Missgunst des Volkes gegenüber Maria und Joseph. Gleiches gilt für die Eltern des Johannes. 
\\~\\
Das Bedürfnis nach Erlösung für das Volk wird v.a.\ bei Matthäus, aber auch bei Lukas wirkungsvoll demonstriert. Beide verarbeiteten das Leid durch die römische Herrschaft in nicht historisch fundierten Geschichten wie dem Kindermord in Bethlehem und die Flucht der Jesusfamilie nach Ägypten. Auch die Volkszählung war ein Zeichen der Unterdrückung. Auch wenn diese Geschichten nur ein ``erfundenes Werk zur Fundierung der theologischen Botschaft'' waren, ist der Kern der römische Unterdrückung dahinter war. Auch wenn nicht unschuldige Säuglinge in Bethlehem getötet wurden, wurden es andere tatsächlich schon. Auch wenn die Familie nicht fliehen musste, wurden andere Israeliten aus ihrer Heimat vertrieben oder mussten fliehen (und Jesus meidete auch gewisse römische Machtzentren auf seiner Reise bis zu seiner Einreise in Jerusalem). Auch wenn es keine Volkszählung gab, gab es später einen Zensus in Judäa nach der Absetzung des Sohnes des Herodes, dass genauso ein Zeichen der Unterdrückung war und für viele eine weitere steuerliche Belastung und die Gefährdung der Lebensgrundlage darstellte. Dem gegenüber wird Jesus in Form des Kindes gestellt, dass konträr gegen Kaiser Augustus steht. Nicht als reicher, mächtiger Feldherr, sonder als friedlicher, armer und politisch unbedeutender Wanderprediger, dessen Anspruch hingegen des von Kaiser Augustus universal und gerechtfertigt war und ist. 

\subsection{Geisteszeugung Jesu und doch nur ein Mensch}
Es ist eine tiefe Diskussion, ob im Urtext Maria als junge Frau oder Jungfrau übersetzt werden sollte. Da die Kindheitsgeschichte an jüdischer Tradition anschließt und in dieser eine (biologisch) jungfräuliche Geburt nie eine Rolle spielt und nie vorhergesagt wurde, ist anzunehmen dass es sich bei der Geisteszeugung um keine biologisch gesehene Jungfrauengeburt handle. Außerdem entstammen Lukas und Matthäus einen hellenistischen Einfluss, in dem Jungfrauengeburten durchaus eine Rolle spielten. Die Jungfräulichkeit ist also eher aus den kulturellen Einfluss von Lukas und Matthäus herzuleiten und sollte nicht biologisch gesehen werden. Leider zieht sich diese Vorstellung bis zum heutigen Tage durch und belegt zuviel Platz in der Weihnachtsgeschichte.
\\~\\
Es ist unwichtig, ob der heilige Geist Maria biologisch geschwängert hat. Wichtig ist, dass der heilige Geist Jesus als seinen Gottessohn ausgewählt hat. Gottessohnschaft bedeutet vollkommene Teilhabe an der Macht Gottes und dem vollen Gehorsam Gott gegenüber. Es geht um eine einzigartige Beziehung Gottes zu seinen menschlichen Sohn. Gott wählte hierfür Marias Sohn aus und der heilige Geist erfüllte sie und dadurch Jesus Geist mit seiner vollen Macht und dieser Geist wohnt in den Körper von Jesus. Wie dieser Prozess biologisch stattfand ist unbedeutend und ein anderes Thema, da Jesus Macht von seinen Geist und seiner Seele ausging und nicht von seinen Körper, ein Körper der menschlicher nicht sein konnte.
\\~\\
Wichtig hingegen ist, dass diese nicht biologisch gesehen jungfräuliche Geburt Jesus noch menschlicher Macht. Jesus hatte Geschwister, eine Mutter und somit eine Familie. Dies ist biblisch belegt und kann entgegen den Willen der Verfechter einer jungfräulichen  Geburt nicht widerlegt werden, da ``adelphos'' immer den Blutsbruder meint. Dies stützt noch mehr die Menschwerdung Jesu, der durch und durch Mensch ist. Und somit wurde Gott durch und durch Mensch. Jesus ist zwar von Gott als seinen Sohn auserwählt und komplett mit ihn erfüllt, was ihn auch eine geistliche Macht gab, biologisch unterschied er sich aber nicht von einen gewöhnlichen Mitmenschen.
\\
Ein zu der Zeit gängiges Verständnis einer Sohnschaft (im königlichen Kontext) war in erster Linie ein rechtliches Verständnis und in zweiter Linie erst ein Verständnis der Blutsverwandtschaft. Ein König bestimmte einen fähigen Mann zu seinen Sohn, der ihn in seiner Vollmacht vertrat und zur rechten des Königs saß (siehe Jörg Zink). Analog wurde Jesus von Gott auserwählt und machte ihn so zu seinen Sohn mit einer geistlichen Vollmacht, die ihn das Predigen und die Wunder ermöglichte. Diese Ansicht mag für viele einen Bruch in ihren Glauben darstellen, jedoch kam der Fokus des Begriffes Sohn auf das Biologische erst nach der Verfassung auf und war zu Zeiten der Verfassung nicht so strikt auf Blutsverwandtschaft getrimmt wie heute. 
Jesus ist wie in einen alten Dogma beschrieben ``Sowohl Mensch als auch Gott'' in Form eines bevollmächtigten Stellvertreter Gottes auf Erden mit hierzu befähigter, geistlicher Macht. Dies bringt uns die menschliche Seite von Jesus enorm viel näher, verändert zugegebenerweise jedoch auch z.B. das Verständnis der Trinität im Vergleich zum gängigen Verständnis.
\\~\\
Seine Menschlichkeit wird auch bei Johannes den Täufer klar. Es ist wahrscheinlich, das Jesus sein Schüler war. Wie von den Engeln angekündigt ebnete Johannes Jesus den Weg. Er schuf nicht nur eine Bewegung und Aufmerksamkeit für die Missstände, er bereitete auch Jesus auf seine Aufgabe vor. Es ist nicht eindeutig geklärt, ob sich Johannes selber seinen Schicksal bewusst war und/oder sich als Wegbereiter verstand, jedoch ist dies egal. Er schulte Jesus, den Menschen der selbst noch lernen und sich auf seinen Weg vorbereiten musste. Nicht, weil seine göttliche Seite geschult werden musste, sondern weil seine menschliche Seite auf seine Bestimmung vorbereitet werden und lernen musste, mit der göttlichen Seite umzugehen. Johannes bereitete ihn, wissentlich oder unwissentlich, darauf vor. Dies alles mündet in der Taufe, die Weihnachtsgeschichte von Markus (in seine Augen wurde der Messias bei der Taufe erst richtig geboren) in der der Heilige Geist die Gottessohnschaft bestätigt und ihn auf seine Mission sendet. Jesus ist ab da ``bereit'' und wird von Schüler Johannes zu einen Wanderprediger, der Johannes übertrifft.

\subsection{Bezeugung der Bestimmung Jesu}
Sowohl das Magnifikat Marias (Marias Lobgesang vor Gott) als auch die Reise zu Elisabeth sind wahrscheinlich nicht historisch. Dies ist jedoch nicht wichtig. Das Magnifikat war wohl zu der Verfassung des Lukas-Evangeliums ein weitläufiges Lied der ersten Gemeinde. Das Lied ist ein Zeugnis der Erfahrungen der Anhänger und zeigt die Wirkung Jesu auf die frühchristliche Gemeinschaft. Dass es ein bekanntes Lied war bezeugt den Konsens der Urgemeinde in der theologischen Botschaft des Liedes. Dass sich Lukas zutraut, das Magnifikat in den Mund von Maria zu legen, vor der Lukas offensichtlich viel Respekt hatte, bezeugt die theologisch Wahrheit noch mehr. Das Magnifikat ist die Bezeugung von den Wirken Jesu und dessen Sprengkraft. Das Lukas Maria das Magnifikat in den Mund legt, mindert nichts an der Wahrheit der theologischen Botschaft des Magnifikat, dass unendliches Vertrauen in Gott und die Veränderung der Welt aus den Glauben heraus fordert und bezeugt.
\\ 
Gleiches gilt für die Reise zu Elisabeth. Nur, dass sie nicht stattgefunden hat ändert nichts an der Wahrheit der Botschaft, dass Elisabeth und stellvertretend für sie die ersten Anhänger durch den Heiligen Geist die Begeisterung ihres Kindes sowie Jesus als den Messias erkennen können. Beides sind als Bezeugungen für die Gottessohnschaft Jesu und dessen Wirken zu verstehen. Lukas nimmt hier schon das Ergebnis des Wirken Jesu vorweg. 
\\~\\
Gleiches gilt für die heiligen drei Könige, die als eine Gruppe unbekannter Größe aus ``Magiern'' (also Heiden) in der Bibel dargestellt werden. In Anbetracht des Kontextes wird Herodes sie nicht unbeobachtet ziehen lassen, wenn ein neuer König ihn seine Macht strittig machen soll. Er setzte seine Macht und Spionageapparat für viel weniger ein, dann hätte er ohne Zweifel bei der genannten Überlieferung jeden Schritt penibel verfolgt. Direkt damit im Zusammenhang ist der Kindermord in Bethlehem und die Flucht bzw.\ Verfolgung der Jesusfamilie. In Anbetracht des Kontextes hätte Herodes sicherlich die 10--15 Kinder hinrichten lassen und die Familie gesucht, wenn die Geschichte wahr wäre. Jeder einzelne Aspekt der Geschichte fand nicht statt, jedoch greift dies das Ergebnis Jesu Wirken vorweg: Jesus bringt Menschen aus allen Völkern und ethnischen Hintergründen zu Gott und ist für alle gekommen. Aber wir können viel aus der Geschichte lernen. Sie zeigt, dass nicht nur Vernunft zu Gott führt, denn die Vernunft der Magier hat sie zu ihrer heidnischen Religion geführt. Ihr Vertrauen in ihr Gefühl trieb sie die ganze Reise zu Jesus hin durch alle Widrigkeiten und ließ sie direkt ohne nachzudenken vor Jesus niederfallen. Eine erfundene Geschichte, die jedoch über die Jahrtausende trotzdem mehrmals stattgefunden hat (vgl.\ das Buch/Film ``Der Fall Jesus'').

\subsection{Eine Geburt in Armut, Elend um am Rande der Gesellschaft}
Das Jesus in Bethlehem geboren sein soll, ist historisch fragwürdig. Es kann nicht eindeutig geklärt werden, ob die Geburt nun wirklich in Nazareth oder Bethlehem stattfand, jedoch ist die Volkszählung faktisch nicht der Grund des Aufenthaltes, da 
\begin{itemize}
\item[a)]eine solche Volkszählung nicht sattfand und
\item[b)]der Zensus unter Quirinius im Wohnort vorgenommen wurde, da man dort seine Seer und Abgaben zahlte und Joseph garantiert aufgrund der Opferung von Tauben anstatt von Lamm kein Besitz außerhalb von Nazareth gehabt haben konnte.
\end{itemize}
Auch ist fragwürdig, warum eine hochschwangere Frau eine solche gefährliche und anstrengende Reise unternehmen sollte. Jedoch ist die Historizität des Geburtsortes nicht wichtig. Wie bereits zuvor erörtert dient dies zur Legitimation, dass Jesus der Messias ist. Wahrscheinlich wurde er in Nazareth geboren, jedoch sind die Umstände die Gleichen. Jesus wurde unspektakulär in Armut geboren. Behütet durch seine Familie, aber nicht prunkvoll in einen Palast, wo es alle Welt direkt erfährt. Es war eine ``unbedeutende'' Geburt in die Armut hinein. Joseph durfte aufgrund seiner Armut anstatt eines Lammes zwei Tauben für die Geburt Jesu opfern, was nur Armen gestattet wird. Jesus Leben begann am Rande der Gesellschaft und schon dort wird der Fokus des Wirken Jesu auf die verstoßenen und Verlierer der Gesellschaft deutlich. Natürlich erfüllt dies nicht die Prophetie von Micha, doch ist dies eine Auslegungssache, v.a.\ ob Micha Bethlehem als Geburtsort oder Abstammungsort der Familie meint.
\\
Ob Jesus wirklich von Hirten besucht wurde, kann mittlerweile nicht mehr geklärt werden. Jedoch betont auch dies, dass Gott sich auf die niederste Ebene der Gesellschaft herabließ, sodass Hirten, sowohl sozial als auch religiös angefeindet in der Gesellschaft, die ersten waren, die den Messias sehen durften. Ihr Verhalten ist bemerkenswert. Sie ließen alles auf die Verheißung hin stehen und begegneten Jesus. Sie gingen recht rasch wieder zurück in ihren Alltag und preisten und lobten den Herrn. 
\\~\\
Dass (nicht erfundene) Motiv der Armut Jesu und sein Zugehen auf Randgruppen zieht sich durch die ganze Weihnachtsgeschichte als auch die Evangelien. Gott begab sich von seinen himmlischen Thron so tief in den Dreck, um uns zu erretten. Daher scheint neben der offensichtlichen  Legitimation des Anspruches auf die Gottessohnschaft die Geburt in die Armut, das Elend und dem Rande der Gesellschaft hinein und damit der Fokus der Heilmission Jesu auf diese Gruppen zu sein.

\subsection{Fazit}
Der offensichtliche Kernpunkt ist die Menschwerdung Gottes und somit die Gottessohnschaft. Dabei gab es keine Jungfrauengeburt, sondern eine reine ``geistliche'' Zeugung, bei der Jesus von Gott als seinen Sohn bevollmächtigt und hierfür mit seinen bekannten Fähigkeiten bestückt wird, aber mit keinen biologischen bzw.\ göttlichen Körper. Auch wird schon das Ziel Jesu vorgegriffen: Die Menschheit -und nicht nur das Judentum alleine- zu erlösen und zu befreien (mehr dazu später).
\\~\\
Ein weiterer zentraler Aspekt ist die Geburt in die Armut das Elend der Gesellschaft herein. Die Geburt war nicht pompös, edel oder königlich in aller Öffentlichkeit, sondern ein ``unbedeutendes'' Event, in den die Armen und Verstoßenen als erste und einzige das Jesuskind sehen durften.
\\~\\
Die Reise zu Elisabeth, der Stammbaum, die heiligen drei Könige und der Kindermord in Bethlehem als auch die Flucht fanden wahrscheinlich nicht statt, jedoch stimmt die Botschaft dahinter und wird später im Leben von Jesus Christus bestätigt. Dies sind also Projektionen späterer Ereignisse und dient primär der Legitimation der Gottessohnschaft Jesu. 
\\ 
Generell kann die Kindheitsgeschichte Jesu als Ouvertüren der Evangelien und das Wirken Jesu gesehen werden.

\subsection{Feiern des Weihnachtsfestes im heutigen Kontext}
Das Weihnachtsfest ist ein Menschenwerk und hat keine biblischen Riten o.ä.. Es ist ein Gedenkfest und die inhaltlichen Themen zum darauf Zurückbesinnen können den Fazit entnommen werden. Jedoch kann sicher gesagt werden, dass mit Weihnachten das Kirchenjahr beginnen soll, sofern Jesu Wirken als ein Rahmen für das Kirchenjahr angenommen wird, da Jesu Geburt offensichtlich der Anfang seines Wirkens ist. Abseits dem theologischen Auseinandersetzen und Besinnen kann, was den Rahmen, Rituale und die Strukturen des Festes angeht, folgendes gesagt werden:
\\~\\
Weihnachten sollte genau wie die Geburt damals \textit{kein} pompösen, überzogenes und überhobenes Fest sein. Es sollte sich auf das offensichtliche berufen werden: Das Kind in der Krippe. Das Christuskind war umhüllt von armen Gestalten und Verstoßenen der Gesellschaft. Deshalb sollte zu der Zeit auf die Armen und Ausgestoßene, im Volksmund die ``Verlierer der Gesellschaft'', geschaut werden und reflektieren, wie man diesen ganz aktuell und generell helfen kann. Auch kann man auf die symbolische Armut der Mitmenschen geschaut werden. Wie zuvor beim Punkt Advent erörtert sollte man die Dinge geben, die man selbst im Überfluss hat und wovon ein anderer benötigt, sei es materiell oder immateriell. Und dies kann auch durchaus Geschenke für die materiell Armen geschehen, zu welchen auch so ziemlich alle Kinder gehören, die durch den Reichtum ihrer Eltern beschenkt werden (im wortwörtlichen als auch übertragenen Sinne).
\\~\\
Das Christuskind war umgeben und behütet von seiner Familie, auch später in seiner Kindheit. Das Weihnachtsfest sollte daher ein Fest für und mit der Familie sein, sowohl die Blutsfamilie als auch die geistige Familie. In Verbindung mit zuvor genannten kann man sich auch mit Freunden treffen, die von Ferner kommen und/oder mit denen man (aufgrund der vorherrschenden geographischen Distanz) im Jahr über weniger Zeit verbringt. Dabei sollte die Liebe der Familie zueinander im Vordergrund stehen wie die Liebe von Jesus Eltern zu ihn. Materielle Dinge sowie der Kommerz sollten im Hintergrund rücken und gehören nicht zum christlichen, sondern zum weltlichen Weihnachtsfest, was jedoch nicht heißt, dass es auch in einen begrenzten Rahmen im christlichen Weihnachtsfest Raum finden kann, wenn es den eigentlichen Kernpunkten zugute führt.  
\\~\\
Man kann sich auf auf die Erlösung und das Ziel der Heilmission Jesu berufen, die mit Jesus Christus auf die Erde kommt. Hierzu zählt u.a.\ auch das Kreuz und die Endzeit, jedoch haben diese Themen später im Kirchenjahr auch ihren Platz, weswegen sie nicht zu sehr Platz einnehmen sollten. Wichtiger als der bevorstehende Weg ist immernoch die Geburt bei Weihnachten.

\subsection{Kindheitseschichte}
Die Kindheitsgeschichte von 12-jährigen Jesus gibt es nicht allzuviel neues zur Geburtsgeschichte zu erzählen, bevor wir zur nächsten großen Kapitel (die Taufe) kommen. Der Vollständigkeit halber ein paar Anmerkungen zu diesen Text: Er kommt nur in Lukas vor und ist der einzige Kindheitstext im Kanon. In den ist Jesus sich bereits seiner Gottessohnschaft bewusst, jedoch auch, dass sein Zeit noch nicht gekommen ist und er noch heranwachsen muss.
\\~\\
Hier wird wieder die Dualität zwischen Jesus als Mensch und als Gottes Sohn aufgezeigt. Einerseits muss er als Mensch lernen. Er löchert die Schriftgelehrten mit Fragen und geht danach wieder zurück nach Nazareth, um heranzuwachsen. Die ganze Zeit ist er seiner Familie untergestellt, trotz seiner höchst göttlichen Berufung. Andererseits ist er bevollmächtigt durch Gottes Geist. Er als 12 jähriges Kind hat -was theologische Themen betrifft- eine geradezu übernatürliche Auffassungs- und Deutungsgabe. Er ordnet die Antworten der Schriftgelehrten schnell ein und bietet ihnen Paroli und das als junges Kind. Seine geistliche Vollmacht Gottes befähigt ihn dazu. Es ist so, als würde ein 10 jähriger (die geistliche Entwicklung heutiger Menschen ist unterschiedlich zu damals, weil man heute nicht mehr mit 14 Erwachsen ist und ein Arbeitsleben bestreiten muss usw) mit Professoren an der Uni auf Augenhöhe diskutieren.