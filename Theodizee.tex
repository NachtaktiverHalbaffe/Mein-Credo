% !TEX root = Lehren__von_Jesus_Christus.tex
\chapter{Das Böse \& Theodizee-Problem}\label{sec:Theodizee}
Der Begriff des ``Bösen`` und des dadurch verursachten Leids ist breit interpretier- und definierbar. Deshalb wird er im Rahmen der folgenden Erläuterungen eingegrenzt. Das „Böse” ist eine sprachliche Wertung, die Orientierung in Moral und Ethik schafft. Dabei lässt sich der Begriff nicht abstrakt definieren, sondern er ist immer konkret. Er wird erlebt, erfahren und erlitten. Es widerfährt dem einzelnen Menschen oder einem Kollektiv. Das Böse lässt sich durch seine Wirkung beschreiben: Es schädigt Menschen und zerstört Leben. Die Wirkung des Bösen ist also erfahrbares Leid. Somit lässt sich das Böse als die Erfahrungen und Sachverhalte definieren, die negativ erfahrbares Leid verursachen. Dabei kann das Böse nicht ohne das Leid existieren und umgekehrt. Wenn das Leid erklärt oder gerechtfertigt wird, wird somit auch implizit das Böse erschlossen – und umgekehrt.\\

\begin{wrapfigure}[20]{l}{0.45\linewidth}
    \begin{center}
        \includegraphics[scale=0.13]{theodizee}
        \caption{Die 3 Übel}\label{fig:theodizee}
    \end{center}
\end{wrapfigure}

Da das Thema Leid im Rahmen des Theodizee-Problems in der Theologie-Welt schon ausführlich betrachtet wurde, soll es im Folgenden näher untersucht werden. Leid lässt sich in drei Kategorien unterteilen (siehe Abbildung \ref{fig:theodizee}):
\begin{itemize}
    \item Malum Metaphysicum: Das Böse, dass durch die Unvollkommene und sterbliche Natur des Menschen verursacht wird
    \item Malum morale: Das Böse, das durch das Handeln eines konkreten Menschen verursacht wird
    \item Malum physicum: Das Leid, dass durch das in der Natur vorhandene Übel ausgelöst wird z.B.\ Naturkatastrophen
\end{itemize}
Die erste Kategorie wird meistens damit gerechtfertigt bzw. erklärt, dass etwas von Gott Geschaffenes von diesen
verschieden und somit unvollkommen sein muss. Die beiden anderen Kategorien sind der Kerngegenstand der Diskussion. Die Theodizeefrage beschäftigt sich mit dem Konflikt zwischen dem Gottesbild eines allmächtigen, allwissenden und barmherzigen Gottes und der Erfahrung sinnlosen Leids in der Welt. Für viele stellt dies einen Widerspruch zum christlichen Glauben dar und wird auch als ``Fels der Atheisten`` bezeichnet. Dabei geht es nicht um eine Rechtfertigung oder Erklärung Gottes, sondern darum, wie der persönliche Glaube trotz des Theodizee-Problems Bestand haben kann. Daher werden nachfolgend Argumente geliefert, die einen Glauben an einen barmherzigen, allmächtigen Gott angesichts des Leids auf der Welt ermöglichen. Es handelt sich also nicht um einen Beweis, sondern um eine Verteidigung des Glaubens.\\

\section{Wer ist Schuld am Bösen?}
Ein erster Schritt zur Lösungsfindung des Theodizee-Problems ist die Befassung mit der Frage, wer für das Böse und somit für das Leid verantwortlich ist. Es geht also nicht darum, wer das Böse ausübt und somit Leid verursacht. Diese Frage wurde bereits mit den drei Übeln ``malum metaphysicum``, ``malum morale`` und ``malum physicum`` geklärt. Es geht im wahrsten Sinne des Wortes darum, die „Wurzel des Bösen” zu finden.

\subsection{Der Mensch ist Schuld}
Ein Erklärungsansatz ist, die Schuld am Bösen dem Menschen und seiner Natur zuzuschreiben. Demnach trägt Adam ein böses Herz. In der Erbsündenlehre versucht Paulus, das Böse in der menschlichen Natur auf die Erbsünde und das Böse des Menschen zurückzuführen, indem er die Abstammung von Adam und Eva heranzieht. Dabei ist wichtig anzumerken, dass es Paulus nicht um eine biologische Abstammung geht, sondern um die Abstammung des menschlichen Wesens. Ein wichtiges Wesensmerkmal des Bösen ist laut Paulus, dass es nicht durch die Tora überwunden werden kann, sondern ein Verhängnis des Menschen ist. \\

Zusammengefasst führt dieser Erklärungsansatz das Böse auf die Unvollkommenheit der Menschen zurück. Dies ist plausibel und deckt sich mit den Erfahrungswerten der Menschheitsgeschichte. Die Erbsünde als Erklärung für die Herkunft wird im entsprechenden Kapitel zum Thema Sünde genauer erläutert. Ein entscheidender Schwachpunkt dieses Erklärungsansatzes ist jedoch, dass die Schuld nicht beim Menschen zu verorten ist, da er nichts für sein Wesen kann. Somit ist entweder Gott als Schöpfer implizit schuld am Bösen oder der Teufel muss als personifizierte Figur des Bösen herhalten. Diese beiden Ansätze werden nachfolgend näher erläutert.

\subsection{Der Teufel ist Schuld}
Ein Erklärungsansatz ist, die Schuld am Bösen dem Teufel und den gefallenen Engeln zuzuschreiben. Laut einer außerbiblischen Überlieferung hegt der Teufel einen Groll gegen die Menschen, da er sich geweigert hatte, Adam bzw. die Menschen als Ebenbild Gottes anzubeten. Als Konsequenz wurde er von Gott verbannt. Laut der Offenbarung des Johannes rebellierte der Satan anschließend gegen Gott, verlor den Kampf und fiel für eine gewisse Zeit auf die Erde, um bis zum Jüngsten Gericht zu verweilen und dann endgültig besiegt zu werden. Nach neutestamentlicher Sicht wurde der Satan somit zum Herrscher der Welt, der dann von Jesus besiegt wurde. Da der Satan und die Dämonen nichts mehr zu verlieren haben, versuchen sie, den Menschen das Böse beizubringen. \\

Wie in Kapitel  \ref{sec:Teufel} erklärt, existiert der Teufel nicht und ist daher nur ein überdramatisierter Erklärungsversuch für das Böse in der Welt. Die Figur des Teufels diente lediglich dazu, das Böse zu erklären und aufzuzeigen, dass Gott das Böse durch Jesus besiegt hat. Darauf aufbauend entwickelte die Alte Kirche einen Teufelkult und lenkte die Aufmerksamkeit für Jahrtausende von der eigentlichen Frage ab. Mit der Abschaffung des Teufels entfällt dieser als Sündenbock für das Böse und der Mensch muss sich der Tatsache stellen, dass das Böse von ihm ausgeht (siehe Theodizeefrage). Damit kommen aber auch eine finale Wahrheit und ein Paradoxon zum Vorschein, die im nächsten Abschnitt erläutert werden. Gott ist schuld.

\subsection{Gott ist Schuld}
Die Erkenntnis, dass Gott als Schöpfer des Menschen und somit implizit als Ursprungsquelle alles erfahrbaren Bösen Schuld am Bösen ist, führt zu einem Paradoxon. In diesem kann die Charakteristik eines allmächtigen, allwissenden, barmherzigen und souveränen Gottes nicht mit der Schuld am Bösen auf einem allgemeinen Level erklärt werden. Es gibt jedoch Erklärungsversuche, die sich einer Antwort annähern, die für einige Menschen genügen könnte (siehe Theodizeefrage). Sobald man jedoch versucht, die Schuld am Bösen auf eine Weise zu erklären, die Gott von seiner Schuld entbindet, suggeriert man, mehr von der Welt und ihrer Geschichte zu wissen als Gott selbst, und erhebt sich somit über den allmächtigen König. \textbf{Dies ist die Prämisse, mit der wir in die Theodizeefrage gehen müssen: Die Schuld am Leid trägt ganz und allein Gott.}
    
\section{Die Anwort auf das Malum morale: Free Will Defense} \label{free_will_defense}
In der zeitgenössischen Theologie wird zur Beantwortung der Theodizeefrage die Willensfreiheit des Menschen als zentrales Element herangezogen. Demnach braucht es für wahre, wechselseitige Liebe zwischen Gott und den Menschen die Freiheit beider Parteien. Ohne diese Freiheit ist keine authentische Beziehung zwischen Gott und Menschen möglich. Wenn Gott also einen Menschen für sich gewinnen will, bleibt ihm keine andere Wahl, als die Freiheit der Menschen zu akzeptieren. Dies lässt jedoch auch den Missbrauch der Freiheit zu, der zu Leid durch andere Menschen führt. Um dieses Argument zu stützen, müssen fünf wichtige Kernpunkte verteidigt werden.
\begin{enumerate}
	\item Die Freiheit des Menschen ist für sein moralisches Handeln erforderlich. Sie kann weder mit Mitteln der theoretischen Vernunft noch mit Mitteln der empirischen Wissenschaft widerlegt (oder bestätigt) werden (siehe Kapitel \ref{Willensfreiheit})
	\item Die Existenz von Personen, die in Freiheit das moralisch Wichtige wählen können, ist wertvoller als die Existenz von Personen, deren Handeln durchgängig vorhersagbar und vorherbestimmt ist (siehe Kapitel \ref{Werthaftigkeit_Freiheit})
	\item Die Freiheit, die man schenkt, beinhaltet auch die Möglichkeit, sich für das Falsche zu entscheiden (siehe Kapitel \ref{Begrenzung_Freiheit})
	\item Jede Person, die vom Leiden betroffen ist, kann in Bezug auf die eigene Lebensgeschichte zu der Einschätzung kommen, dass der positive Wert der Freiheit das Leiden aufwiegt (siehe Kapitel \ref{Preis})
	\item Es besteht die berechtigte Hoffnung, dass am Ende ihres Lebens jede Person trotz aller erlittenen Leiden Ja zu ihrem Leben sagen wird (siehe Kapitel \ref{ende_leben})
\end{enumerate}

Diese Kernpunkte lassen sich nicht vollumfänglich beweisen. Das ist auch nicht das Ziel. Sie lassen sich jedoch so verteidigen, dass der Glaube an einen barmherzigen, allwissenden und allmächtigen Gott angesichts des Leids dennoch plausibel erscheint. In den nachfolgenden Abschnitten sollen entsprechende Verteidigungsansätze präsentiert werden.

\subsection{Verteidigung der Willensfreiheit des Menschen} \label{Willensfreiheit}
Die Freiheit des Menschen bildet die grundlegende Voraussetzung für moralisches Handeln. Moral setzt notwendig voraus, dass der Handelnde zwischen verschiedenen Handlungsoptionen wählen kann. Nur wenn ein Mensch die Möglichkeit hat, anders zu handeln, kann man ihm Verantwortung zuschreiben. Ohne Freiheit wären Lob und Tadel, Schuld und Verdienst bedeutungslos, da alle Handlungen lediglich kausal determiniert wären. Freiheit ist daher nicht nur ein metaphysisches Postulat, sondern ein praktischer Grundbegriff der Ethik.\\

Diese Freiheit lässt sich jedoch nicht durch theoretische Vernunft nachweisen. Der Versuch, Freiheit auf logischem oder metaphysischem Wege zu beweisen, stößt an Grenzen, da sich der Mensch als Teil der Natur auch den Gesetzmäßigkeiten von Ursache und Wirkung unterworfen sieht. Ebenso wenig kann die empirische Wissenschaft Freiheit widerlegen oder bestätigen: Sie untersucht Handlungen und Entscheidungen zwar auf ihre neuronalen und psychologischen Bedingungen hin, doch diese erklären lediglich die Bedingungen des Handelns, nicht die Möglichkeit freier Wahl selbst. Empirische Daten erfassen Ursachen, aber nicht die normative Dimension, die moralische Verantwortung erst ermöglicht.\\

Stattdessen zeigt sich Freiheit im praktischen Vollzug. In moralischen Entscheidungssituationen erlebt sich der Mensch als frei und verantwortlich. Dieses praktische Bewusstsein der Freiheit ist nicht Ergebnis einer theoretischen Ableitung, sondern Bedingung der Möglichkeit moralischen Urteilens überhaupt. Schon Kant hat darauf hingewiesen, dass Freiheit als Postulat der praktischen Vernunft unverzichtbar ist: Damit moralische Gesetze sinnvoll sein können, muss der Mensch als frei gedacht werden.\\

Somit erweist sich die Freiheit des Menschen nicht als empirisch beweisbare Tatsache, sondern als notwendige Voraussetzung des moralischen Handelns. Weder Theorie noch Wissenschaft können sie aufheben, ohne zugleich die Grundlage ethischer Praxis zu untergraben. Freiheit bleibt damit ein unverzichtbares Postulat, das den Raum eröffnet, in dem moralisches Handeln und Verantwortung überhaupt erst denkbar werden.

\subsection{Verteidigung der Werthaftigkeit der Freiheit} \label{Werthaftigkeit_Freiheit}
Der Fokus liegt hier nicht auf der Frage, ob wir in einer Welt ohne Freiheit, aber ohne Leid leben wollen. Wenn wir uns in einer solchen Welt befänden, wären wir komplett andere Personen. Daher können wir uns auch nicht in diese hineinversetzen. Es geht um die Frage, ob wir lieber willenlos wären, um auf dieser Welt besser zu leben, oder ob wir unser Leben lieber in Freiheit gestalten wollen. Diese Verteidigung kann nicht vollständig auf einem empirisch-argumentativen Ansatz basieren. Daher bedient man sich eines philosophischen Gedankenexperiments:

\begin{displayquote}
Eine Person möchte die Liebe eines Mannes oder einer Frau gewinnen. Ein guter Freund bietet eine Liebespille an, die die andere Person dazu zwingt, einen zu lieben. Man müsste sie der Person nur heimlich verabreichen, damit sie sich unsterblich verliebt.
\end{displayquote}

Entscheidet man sich gegen die Pille, so muss man auch die Freiheit der anderen Person und die Freiheit der Liebe bejahen. Oft wird als Gegenargument angeführt, was passieren würde, wenn der Freund der Person, die er liebt, heimlich die Pille geben würde. Dies würde jedoch nichts an der Werthaftigkeit der Liebe und somit der Freiheit ändern. Es würde nur das Ganze erträglicher machen, wenn man Freiheit will, sie aber nicht bekommen kann. Und das stärkt wiederum indirekt die Werthaftigkeit der Freiheit.

\subsection{Verteidigung, dass Leid nicht ohne Begrenzung der Freiheit reduziert werden kann} \label{Begrenzung_Freiheit}
Ein beliebtes Gegenargument ist, dass Gott die Freiheit so gestalten könnte, dass Menschen nicht böse handeln können. Dies greift jedoch die Definition von Freiheit an. Der Mensch wäre dadurch in seiner Freiheit eingeschränkt, da er nicht mehr frei wählen könnte. Zudem würde ein solcher Gott die Freiheit des Menschen nicht vollständig akzeptieren.\\

Ein zweites Gegenargument ist, dass Gott den Menschen zwar die komplette Willensfreiheit lassen kann, seine Umwelt aber so gestaltet, dass Leid deutlich begrenzt wird. Dieses Argument zielt auf das Malum physicum ab, das in Kapitel \ref{nature_defense}  beantwortet wird. Was viele jedoch in diesem Zusammenhang auch übersehen: Die Endlichkeit des irdischen Lebens stellt bereits eine Begrenzung des Leidens dar, ohne die Freiheit des Menschen einzuschränken. Dies ist jedoch ein gefährlicher Gedanke, da er dazu verleitet, den Freitod als Ausweg aus dem Leid zu wählen.

\subsection{Verteididung, dass die Freiheit dem Preis wert ist} \label{Preis}
Diese Frage, ob der Preis, den man für die Freiheit zahlen muss, das erlittene Leid wert ist, kann nur von der betroffenen Person selbst beantwortet werden. Andernfalls müsste man die verschiedenen Leiden verschiedener Personen aufwiegen, was schlichtweg falsch ist und überhaupt nicht möglich ist. Es gab Menschen in Auschwitz, die dort trotzdem Hoffnung gefunden haben und mit einer lebensbejahenden Einstellung gestorben sind. Und es gibt das Gegenteil. Diese Perspektiven gegeneinander aufzuwiegen, ist unmoralisch und verführt dazu, das Leid von Personen gegeneinander abzuwägen und somit abzuwerten. \\

Eine wichtige Prämisse dieser Frage ist jedoch, dass sie die Existenz der Welt und des Menschen infrage stellt. Wenn man davon ausgeht, dass Gottes Schöpfung der Welt und des Menschen mit der Notwendigkeit der Freiheit des Menschen und somit des Leids durch Menschen und Natur einhergeht, könnte man sagen, dass Gott die Schöpfung einfach hätte sein lassen können. \\

Unter diesen Umständen wird umso klarer, warum nur die Betroffenen selbst entscheiden können, ob ihnen die Freiheit diesen Preis wert ist. Denn davon hängt nicht nur das Schicksal, sondern auch die Existenz der Einzelperson ab. Da es Personen gibt, die die Frage nach dem Wert des Preises mit ``Nein`` beantworten, gibt es nur einen Weg, um die Möglichkeit einer Zustimmung aller Menschen zu dieser Frage zu ermöglichen:
\begin{enumerate}
	\item Es muss nachgewiesen werden, dass ein aktuelles „Nein” zu dieser Frage grundsätzlich revidierbar ist. Das bedeutet konkret, dass eine Person die Frage heute mit „Nein” beantwortet, ihre Meinung aber irgendwann in der Zukunft ändern kann.
	\item Das Argumentationsziel ist dahingehend begrenzt, dass es lediglich darum geht, die Möglichkeit einer Ja-Antwort für alle Menschen offenzuhalten. Eine positive Antwort ist jedoch nicht das Ziel.
	\item Es muss nachgewiesen werden, warum es rational ist, die Hoffnung zu haben, dass alle Menschen zu einem ``Ja`` finden können.
\end{enumerate}

Im folgenden Kapitel wird eine Möglichkeit vorgestellt, wie sich diese Punkte argumentativ belegen lassen.

\subsection{Verteididung, dass eine Person am Ende ihrers Lebens trotz des Leids Ja zu ihrem Leben sagen wird} \label{ende_leben}
Die Argumentation hat folgende Struktur:
\begin{itemize}
	\item[1'] Es ist logisch vertretbar, zu hoffen, dass alle Menschen trotz aller Leiden Ja zu ihrer Existenz und somit zu ihrer Willensfreiheit sagen. Damit akzeptieren sie faktisch das Leiden als Preis für Freiheit und Liebe (konkretisierung von Kapitel \ref{Preis})
	\item[1e] Es gibt Menschen, die aus guten Gründen ein Ja zum Dasein verweigern und diese Weigerung bis zum Tod als endgültig betrachten
	\item[1z] 1' und 1e sind vereinbar, wenn Folgendes gilt: Die Verweigerung aus 1e ist prinzipiell revidierbar. Auch nach dem Tod ist eine solche Revision bzw. das Offenbarwerden einer solchen Revision denkbar
	\item[2'] Der christliche Glaube an das Jenseits lässt sich so formulieren, dass die Hoffnung auf eine solche Revision rational begründet werden kann. Konkret bedeutet dies, dass es eine Theologie gibt, in der Menschen definitiv zu einem Ja bewegt werden können.
\end{itemize}

Der erste Teil von 1z kann direkt bestätigt werden, da wir immer wieder erleben, dass scheinbare Lebensentscheidungen revidiert werden, beispielsweise eine Scheidung trotz des Ja-Wortes zur lebenslangen Ehe. Zudem bestätigen alltägliche Erfahrungen diese Behauptung, dass sich jede achso feste Überzeugung irgendwann ändern kann. Der zweite Teil von 1z kann beispielsweise mit einer Begegnung mit Jesus Christus im Sterbeprozess verbunden werden. Es wäre aber auch eine Offenbarung im Jenseits möglich, sofern die eigene Theologie dies zulässt.\\

Die Annahme von 2 lässt sich durch den Glauben begründen, dass die Begegnung mit Christus die Begegnung mit einer in einer Person verkörperten vollkommenen Wahrheit, Freiheit und Liebe ist. Dies ist eines der christlichen Grundverständnisse und somit vertretbar. Als Gegenargument könnte man anführen, dass hierbei theologische Prämissen, d.h. religiöse Glaubenssätze, verwendet werden, die in einer Diskussion über die Theodizee nichts zu suchen haben, da es hierbei auch um die Existenz Gottes geht. Dem kann jedoch entgegnet werden, dass die Theodizee-Debatte unter der Prämisse geführt wird, dass Gott existiert. Es geht schließlich darum, warum ein guter Gott Leid zulässt. Um diese Frage stellen zu können, muss die Existenz Gottes vorausgesetzt werden. Daher ist es nur fair, wenn auch theologische Prämissen zugrunde gelegt werden.

\section{Die Antwort auf das Malum physicum: Nature Defense} \label{nature_defense}
Ein häufiges Argument ist, dass Gott die Welt und ihre Naturgesetze so schaffen konnte, dass daraus weniger Leid entsteht. Dem entgegnet Hick: Erst die Verschiedenheit der Schöpfung ermöglicht Freiheit und einen eigenen Stand vor Gott. Zudem muss die Welt so geschaffen sein, dass sie sowohl atheistisch als auch religiös gesehen werden kann, um die Freiheit des Menschen zu gewährleisten (siehe Kapitel \ref{free_will_defense}). Wäre die Welt so gestaltet, dass weniger Leid durch Menschen und Naturgesetze entsteht, könnte der Mensch keine genügend kognitive Distanz zu Gott aufbauen. Dies hätte zur Folge, dass die Freiheit des Menschen eingeschränkt wäre und die Liebe nicht mehr komplett auf Vetrauen beruhen würde. 

\section{Das Gift in der Diskussion um Theodizee}
Der zuvor genannte, theoretische Verteidungsansatz zu einen Glauben an einen barmherzigen, allmächtigen und allwissenden Gott birgt eine riesige Gefahr: Es ermöglicht es, dem Leid einen Sinn zu geben. Jedoch darf und kann moralisch nur der Leidende seinen Leid einen Sinn geben. Daher wäre eine allgemeine Sinnesverleihung des Leids mithilfe der zuvor genannten Verteidung tiefst unmoralisch und pervers. So könnte man nämlich bei Unbedachtheit z.B. Ausschwitz einen Sinn geben, welcher definitiv nicht existiert. Daher ist folgendes zu beachten:
\begin{itemize}
	\item Die Verteidung versucht nur den Widerspruch des Theodizeeproblems zu lösen. Diese kann auch gerechtfertigt sein, denn oft fragen die Leidenden selbst nach dem \glqq Warum?\grqq{}. Aber die theoretische Anwort soll nicht zur Rechtfertigung Gottes führen, sondern nur die Problemlage so stabiliseren, dass der Leidende sich mit seinen Nöten und Ängsten an Gott wendet und sich von ihm überzeugen lässt. Deshalb braucht es eine Erweiterung um eine praktische Handlungsorientierung. Diese wird praktische Theologie genannt.
	\item Wenn man den atheistischen Standpunkt einnimmt und aus Protest gegen das Böse einfach Gottes Existenz aus der Welt oder der Argumentation rausstreicht, dann hat man nur eine böse Welt ohne Hoffnung auf Besserung. Dieser Standpunkt ist also ebenso Gift in einer Leidsituation wie die theoretische Anwort. Zudem bietet der Atheismus weder Bewältigungsstrategien als auch keine Antworten auf die Leidfrage.
	\item Die Existenz Gottes gibt also der Leidthematik wieder eine moralische Komponente: Durch die Schuld Gottes am Leid gibt es eine Entität, welche man anklagen kann. Gegen die man protestieren kann. Ohne diese Entität wäre der Protest sinnlos, da es nur im Universum verschallt. Und diese Anklage ist sowohl ein entscheidender Bewältigungsmechanismus als auch ein Mechanismus gegen die Relativierung des Leids, da es so nicht vergessen werden kann und von jemanden wahrgenommen wird.
\end{itemize}

Elie Wiesel bringt es im Drama ``Der Prozess von Schamgorod`` auf den Punkt:
\begin{displayquote}
Juden machen in einen osteuropäischen Dorf Gott den Prozess, weil er sie angeblich in ihren Leidensgeschichte im Stich gelassen hatte. Der Prozess findet in einen Wirtshaus statt, während sich die Antisemiten zusammenrotten und die Juden langsam umbringen. Richter, Ankläger und Zeugen sind schnell gefunden, nur kein Verteidiger für Gott. Da kommt ein Fremder rein und vernimmt die Verteidigung. Er ist so gut, dass er am Ende allen Anklagen den Wind aus dem Segel nimmt und die Menschen  davon abhält, sich weiter bei Gott zu beschweren und sich mit ihm auseinanderzusetzen. Dieser Fremde erweist sich am Ende als Teufel und keine der Juden setzte sich mehr mit Gott auseinander. Gott geriet in Vergessenheit.
\end{displayquote}

\section{Wie gehen wir mit dem Bösen um?}
Gott ist aber nicht nur Schuld am Bösen, sondern auch derjenige, der dieses in allen seine Erscheinungsformen besiegen
und uns somit hindurchhelfen kann. So konnte Jesus zur Zeit seines Wirkens nur Dämonen und somit im übertragenen Sinne
das Böse aus den Menschen austrieben, weil Gott diese besiegt hat. Hiermit wird Gott uns nicht nur als Schuldiger,
sondern auch als Erlösung von den Bösen geboten. Hiermit besteht auch ein universaler Weg für die Überwindung des
Bösen, falls der Mensch dies nicht selber schafft: Vor Gott kommen und darauf Vertrauen, dass er das Böse
stellvertretend für uns überwindet oder durch die Auswirkungen hindurchhilft. Und falls wir nicht die Erlösung
erfahren, die wir uns wünschen, dann bleibt am Ende immernoch die Klage, wie es schon die Menschen in den Psalmen
gemacht haben und welche Gott aushält (muss).


\subsection{Theodizeesensible Rede vom Handeln Gottes}
Die theodizeesensible Rede von Gottes Handeln vereint konkrete Handlungsorientierung und die theoretische Reflexion im Bezug auf das Theodizee-Thema. Ziel hierbei ist es, achtsam und religiös auf eine verantwortliche Art und Weise über Leid zu reden. Diese zeichnet sich durch folgende Punkte aus:

\begin{itemize}
    \item Worte über Gott dürfen Leid nicht verharmlosen oder schönreden, da diese das Leid bagatellisieren und dadurch versuchen, dem Leid vorschnell Sinn zu geben. Dies könnte Opfer verletzen. \textbf{Red Flags} sind hier Formulierungen wie: \glqq Es war Gottes Wille\grqq{} oder \glqq Alles hat einen Sinn\grqq.
    \item Theodizeesensible Rede hat einen Erinnerungscharakter: Sie darf nicht bereits bei der Zuwendung zu Gott und der Ausrichtung auf dessen Verheißungen die vergangenen Leiden der Welt vergessen. Gott wendet nicht den Blick vom Trümmerhaufen der Geschichte ab. Er will das Zerschlagene neu zusammenfügen, doch der Freiheitsmissbrauch der Menschen macht seine Wiederaufbau-Mühen über den Fortlauf der Geschichte immer wieder zunichte. Wenn Gott das Leid nicht vergisst, dann dürfen wir es auch nicht. Jedes Leid ist real und darf nicht vergessen werden. 
    \item Es muss immer die Hoffnung auf eine innere Verwandlung der Bedeutung der Leidensgeschichte bestehen, die es möglich macht, ohne Ausblendung des eigenen und fremden Leidens \glqq Ja\grqq{} zum Leben zu sagen. Sie betont die \textbf{Offenheit der Zukunft Gottes}: Leid ist nicht das letzte Wort, sondern eingebunden in die Hoffnung auf Erlösung, Heilung und Vollendung.
    \item Das Thema Leid wird eingebunden in Glaubenserfahrungen und Dogmen. Daher geht es dabei immer auch um die Wahrnehmung der Würde des Leidenden, die es absolut zu respektieren gilt.
    \item Man darf Gott anklagen und gegen das Leid protestieren. Klage und Zweifel sind nicht Zeichen mangelnden Glaubens, sondern Ausdruck lebendiger Beziehung zu Gott. In Predigt und Unterricht soll die Rede vom Leid \textbf{biblische Klage-Traditionen} (z.\,B. Psalmen, Hiob, Jesu Kreuzschrei) aufnehmen und so zeigen, dass Klage und Anklage Teil des Glaubens sind.
    \item Der Mensch ist von Gott gewürdigt und befähigt, an der Verwandlung der Welt durch die Kraft der göttlichen Liebe mitzuwirken. Theodizeesensible Rede sucht nach \textbf{Solidarität im Handeln}: Neben der Sprache über Gott ist auch das tätige Mit-Leiden, Zuhören, Begleiten und Helfen ein theologisches Zeugnis. Praktisch bedeutet theodizeesensible Rede auch, \textbf{achtsam mit Metaphern und Bildern} umzugehen: Nicht jedes biblische oder kirchliche Bild eignet sich in Situationen von Leid, da manche trostlos oder verletzend wirken können.  
    \item Theodizeesensible Rede erträgt die Spannung zwischen Gott und dem leidenden Menschen, ohne sie vorschnell aufzulösen. Seelsorgerliche, pädagogische und liturgische Sprache sollte \textbf{offene Fragen zulassen} und Betroffene in ihrer Sprachlosigkeit nicht überfordern. Schweigen kann manchmal angemessener sein als vorschnelles Reden.
    \item Pädagogisch bedeutet theodizeesensible Rede, Kindern, Jugendlichen oder Ratsuchenden nicht fertige Antworten zu geben, sondern \textbf{Frageräume zu öffnen} und eigene Zweifel und Unsicherheiten transparent zu machen.
    \item Schließlich geht es darum, \textbf{das Leid nicht nur individuell, sondern auch gesellschaftlich zu deuten}: Theodizeesensible Rede schließt die Kritik an ungerechten Strukturen und den Aufruf zu Gerechtigkeit und Frieden mit ein.
\end{itemize}

\section{Fazit}
Das Böse, wie es die Menschen erfahren, geht von den Menschen aus. Erklärungsversuche wie die Erbsündentheorie sind hierbei nicht nötig und förderlich, da am Ende unweigerlich die Schuldfrage auf Gott zurückgeführt wird. Grund für das Leid und das Böse ist die Freiheit des Menschen, die von Gott geschenkt wurde. Diese Freiheit ist eine zwingende Grundlage, damit der Mensch wahre Liebe und Vertrauen zu Gott aufbauen kann. Dies beinhaltet auch, dass die Freiheit eine Welt erfordert, die auch Leid beinhaltet, und dass sie dem Menschen die Möglichkeit eröffnet, sie für Böses zu missbrauchen. Gott und Jesus sind jedoch auch diejenigen, die das Böse überwinden können und somit in der Lage sind, uns von dem Bösen zu befreien, an dem wir selbst scheitern und verzweifeln. \\

\noindent\fbox{\parbox{\textwidth}{Das Böse ist mithilfe eines allmächtigen, souveränen und allwissenden Gottes nicht erklärbar. Die Frage
        des Bösen kann nicht Allgemein und zufriedenstellend beantwortet werden. Dementsprechend kann an dieser Frage der
        eigene Glaube zerschellen. Die entscheidende Frage lautet daher: \textbf{Was ist Stärker? Die Zweifel und das Leid
            durch das Böse oder das Vertrauen in Gott, dass er uns in unseren Kampf gegen das Böse hilft und uns befähigt, diesen
            auch zu gewinnen?}}}

\section{Eine eindringliche Geschichte}
Ich möchte das Kapitel nicht wie gewohnt mit einenm Fazit schließen, sondern einer Geschichte von Jürgen Spieß. Diese bietet eine hoffnungsvolle Perspektive auf Leid und kombiniert diese mit der Gnade Gottes durch den Kreuzestod Jesu.