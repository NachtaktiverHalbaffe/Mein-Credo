% !TEX root = Mein Credo.tex
\part{Die Theologie als Wissenschaft}
\chapter{Das Problem der Theologie als Wissenschaft}
Die heutige Theologie ist schwierig zu den konventionellen Wissenschaften einzuordnen. Die Theologen sind keine reine Philosophen, sie sind keine reinen Historiker, sie sind keine reine Sprachwissenschaftler und sie sind keine reinen Ethiker. Doch alles spielt in die Theologie mit rein. Es steht außer Frage, dass die Theologie mit wissenschaftlichen Kriterien und Methoden arbeitet. Und in der Konkurrenz der anderen, großen Geisteswissenschaften muss sie das auch, um sich den Titel der Wissenschaft zu verdienen. Dennoch setzt sich die Theologie mit etwas übernatürlichen auseinander, dass weder wissenschaftlich bewiesen noch widerlegt werden kann. Sie kann genauso wie die Historik mit Fakten arbeiten, die durch z.B. archäologischen Funde belegt werden können. Doch schlussendlich stoßt sie an eine Punkt, der außerhalb der Fähigkeiten einer Wissenschaft liegt. Einen Punkt, indem es nur noch Glauben gibt oder nicht Glauben. Dennoch wird versucht, solche irrationale Gegebenheiten mit rationalen Kriterien und Methoden zu erklären. Die Theologie versucht immer mehr, aus dem Glauben Wissen zu machen und dies ist unmöglich.
\\ 
Die Konsequenz ist, dass Theologen sich immer mehr als Wissenschaftler sehen. Der Glaube an Gott ist sporadisch vorhanden, jedoch gleicht die innere Einstellung mehr die eines Agnostikers. Theologen sind bei ihren Exegesen kritisch eingestellt. Sie wollen eher Gottes Botschaft kritisieren, anstatt diese gründlich zu erschließen. Irrationales wie Prophetie wird versucht zu falsifizieren und Rationales zu verifizieren. 
\\ ~ \\
Ein Beispiel für diese Entwicklung ist die historisch-kritische Forschung. Diese Methodik wird seit dem 20 Jh. angewandt und arbeitet mit rationalen Maßstäben. Nach 60 Jahre historisch-kritischer Forschung stellt sich die Frage, welchen Nutzen die sie für den Glaubenden gebracht hat. Die Antwort ist, dass es kaum einen gab. Wir können zwar den Entstehungsprozess einiger Bücher rekonstruieren, jedoch sind die Rekonstruktionen teilweise sehr umstritten. Sie sucht nie einen Weg, die Ergebnisse stimmig in den Glauben einzufügen. Die Ergebnisse müssen sich nicht in den Glauben einfügen, sonder der Glaube muss sich in den Ergebnissen einfügen. Die historisch-kritische Methode hat systematisch den Glauben zerlegt. Doch das Problem ist nicht die Methode an sich, sondern das Ziel der Anwender. 
\\ ~ \\
Der Glaube verliert an Bedeutung. Die Wissenschaft zerlegt alles in immer kleinere, detailliertere Bestandteile und beurteilt diese. Einzelne Aspekte des Glaubens werden herabgesetzt. Der Glaube wird entwertet. Es ist selbstverständlich, dass dies nicht bei allen Theologen zutrifft, dennoch ist dies ein besorgniserregender Trend.

\section*{Die Grundprinzipien der Theologie als Wissenschaft}
Eta Linnemann, eine ehemalige Vertreterin der historisch-kritischen Forschung, formulierte in ihren Buch \glqq Original oder Fälschung?\grqq{} die Prinzipien der Theologie als Wissenschaft und stellte dieser ihrer Kritik gegenüber.
\\ ~ \\
1. Die Realität Gottes wird im vorne hinein ausgeklammert, da geforscht wird. Als letzte Konsequenz ignoriert man seinen eigenen Glauben und die Existenz Gottes, was in speziellen Thematiken wie der Entstehungsprozesses des Buches Jesaja fatale Folgen haben kann.
\\ ~ \\
2. Nur wissenschaftlich belegbare oder rekonstruierbare Inhalte der Bibel werden als wahr angesehen. Gottes Wort, dass nicht mit den wissenschaftlichen Maßstäben stand halten kann, ist abzuweisen. 
\\ ~ \\
3. Das christliche Glauben und ihre Bibel ist auf eine Vergleichsebene mit den anderen Religionen samt ihrer heiligen Schriften einzuordnen. Diese Methodik beruht auf keiner Tatsache, sonder von der Abwendung vom lebendigen Gott.
\\ ~ \\
4. Die Bedeutung der Bibel als heilige Schrift wird relativiert. Da andere Religionen auch ihre heiligen Schriften haben, hat die Bibel ihr Alleinstellungsrecht verloren und wird mit anderen Schriften gleichgesetzt. Dadurch reduziert man das lebendige Wort Gottes auf ein theologischen Begriff und macht daraus tote Buchstaben, die man dann bei der Verkündigung mithilfe Psychologie, Soziologie, Sozialismus o.ä. wieder Leben einhauchen will. 
\\ ~ \\
5. Es wird unterstellt, dass Bibelwort und Gotteswort nicht ein und dasselbe sind. Erst, wenn der Prediger das Bibelwort als Gotteswort erachtet, wird es dessen Bedeutung zugemessen. Man spielt so die Bibelstellen gegeneinander aus und bauscht Unterschiede zu Unvereinbarkeiten hoch. Die Einheit der Bibel wird angezweifelt. Man glaubt nicht mehr daran, dass sich die Schriften ineinander ergänzen. Sie werden als 3000 Jahre alte schriftstellerische und theologische Erzeugnisse erachtet. Dadurch verlieren sie anscheinend ihren Aktualität. 
\\ ~ \\
6. Die heilige Schrift darf nur für den Privatgebrauch ohne der historisch-kritischen Interpretation ausgelegt haben. Eine verantwortliche Auslegung für dritte z.B. eine Predigt habe nach bestimmten, methodischen Regeln abzulaufen.Der lebendige Heilige Geist wird dadurch leblos und weht nicht mehr dort, wo er wehen will. Wenn zwei Theologen mit dieser Verhaltensweisen anfangen zu debattieren, dann stößt man auf zwei Meinungen. Die Bibellehrer, die das Wort hingegen wörtlich nehmen und im Vertrauen des Heiligen Geistes mitteilen, werden die ein und dieselbe Lehre haben. 
\pagebreak
\\
7. Ein nicht erklärter, aber praktizierter Grundsatz der biblischen Exegese ist, dass das geschriebene Wort in seiner jetzigen Form nicht stattgefunden haben kann. Zurückzuführen sei das auf das System der Universitätstheologie, der die Menschen, die nach Anerkennung trachten, einen Zwang gibt, sich einen Namen machen zu müssen. So gelten Theologiestudenten, die die Neue Urkundenhypothese, die Aufgliederung der Briefe des Paulus in drei Schichten und das Zebedaide Johannes nicht der Verfasser des Johannesevangelium sei nicht kennen oder akzeptieren als bescheuert. Dass die Professoren und Studenten in ein Glaubenskonflikt kommen oder vom Glauben abdriften, sei dann nicht verwunderlich.     
\\ ~ \\
8. Der kritische Verstand entscheidet darüber, was in der Bibel Realität sein kann und was erfunden sei, basiert auf den alltäglichen Erfahrungen eines Menschen. Geistliches wird weltlich beurteilt und die Erfahrungen von Gotteskindern ignoriert. So werden selbst die modernen Wunder, selbst wenn sie wissenschaftlich einwandfrei nachgewiesen sind zur Kenntnis genommen. Stattdessen werden Veröffentlichungen und Berichte dieser Wunder als Erbauungsliteratur abgewertet.
\\ ~ \\
9. Nach ihrem eigenen Selbstverständnis will die historisch-kritische Theologie Hilfe zur Verkündigung des Evangeliums leisten durch eine Bibelauslegung, die wissenschaftlich zuverlässig und objektiv ist. Diese Zuverlässigkeit und Objektivität sind jedoch nicht gegeben.
\\
Zu einen sieht man dies mit dem Umgang mit der Literatur. Laut Theorie müsste man alle Bücher zu einen Thema lesen, bevor man Stellung bezieht. Dies ist jedoch aufgrund der Überflutung mit theologischer Literatur unmöglich. Oft stellt schon die Sprache ein Hindernis dar. Bevor man sich den großen Aufwand auf sich nimmt, eine neue Sprache zu lernen, liest man lieber muttersprachliche Literatur. Auch erweist sich der Zugang zu der Literatur als schwierig, da es durchaus ein viertel Jahr oder länger dauern kann, bevor man die Literatur per Fernleihe ausleihen kann. Daher beschränkt man sich auf der zugänglichen Literatur. Außerdem setzt sich eine fragwürdige Technik durch, ein solches Buch mithilfe einer Anmerkung und nach einer verzerrten Kurzdarstellung abfällig zu beurteilen. Aufgrund der enormen Simplifizierung und Reduzierung wird dieses Werk indiskutabel. Von einer objektiven Forschung kann kaum gesprochen werden.
\\ ~ \\
10. In zunehmenden Maße ist bei jungen Theologen eine sozialistische Unterwanderung festzustellen.  An die Stelle des Heilplans Gottes und die ewige Erlösung durch Jesus Christus sind menschliche Ziele der Weltverbesserung getreten.
\\ ~ \\
Dies alles trifft nicht auf alle Theologen, die die historisch-kritische Theologie betreiben, zu, jedoch weißt dies die existentiellen Probleme der Theologie auf, die zwischen dem schmalen Grat zwischen objektive Wissenschaft und Glauben wandeln. Dabei fällt die Theologie ihren Anspruch, eine Wissenschaft seien zu wollen, selber zum Opfer. Zurecht wird des öfteren gefragt, welche Ergebnisse die historisch-kritische Forschung erzielt habe, die einen persönlich auch im Glauben merklich vorangebracht hat.        

\section{Das Idealbild der Theologie}
Die Theologie sollte eine Wissenschaft sein, die den Glauben an Gott stützt. Sie sollen komplexe, kontroverse oder leicht zu missverstehende Themen aufschlüsseln, auslegen und für den Laien in eine leicht verständliche Weise runter brechen. Dafür sollen wissenschaftliche Methoden wie die Textanalyse angewandt werden. Dabei sollte das Ziel eine Fundierung seines Glaubens sein und nicht -wie heutzutage oft angewandt- als einen absoluten Beweis für seine Glauben. Es gibt eine Grenze zwischen (für uns) Erklärbaren und Unerklärlichen. Dies ist die Grenze zwischen an etwas Glauben oder nicht an etwas Glauben. Ab dieser Grenze ergeben rationale Ansätze und Methoden keinen Sinn und sollten nicht angewendet werden. Das Ergebnis ist, dass die Theologie einen eine Theorie für plausibel oder unplausibel erklärt, sodass man reines Gewissens daran Glauben kann. Die Theologie muss aufhören, sich selbst als Wissenschaft behaupten zu müssen. Theologen sind keine Wissenschaftler im klassischen Sinne, sondern professionelle Erschließer des Glaubens. Ein verharren auf ein rein wissenschaftliches Selbstverständnis führt zwangsläufig in ein Glaubenskonflikt und diese sollten eigentlich von der Theologie gelöst werden. 
\\ ~ \\
Die Theologie kann vor allem in Bezug auf die Bibel nur die menschliche Seite mithilfe von Methodiken erschließen, doch die göttliche Seite lässt sich nicht von solchen banalen Techniken beeinflussen. Das Göttliche an dem lebendigen Wort Gottes lässt sich allein aus reines Übertragen auf sein eigenes Glaubensleben erschließen und Glauben ist eine Tat in dieser Ebene, frei von Wissenschaft, Methodiken und irgendwelchen wissenschaftlichen Verständnissen. Dieser Wechsel zwischen der menschlichen und der göttlichen Ebene lässt nicht wenige Theologen scheitern.
\\ ~ \\
Der richtige Umgang mit den Ergebnissen ist wichtig. Ergebnisse können verschieden ausgelegt werden, sowohl gegen den Glauben als auch für den Glauben. Letzteres sollte offensichtlich der Fall sein. Alleinstehend mögen die Ergebnisse Zweifel aufwerfen, doch zusammen können diese Zweifel in den Einklang des Glaubens gebracht werden. Man kann den Glauben in immer kleinere Bestandteile zerlegen, bis nur noch das kleinste Detail übrig bleibt. Lässt man es in diesen Zustand, dann treten die zuvor beschriebenen Probleme auf. Fügt man jedoch alles wieder zu einen großen zusammen, so negieren sich die Zweifel gegenseitig oder fügen sich alle Ergebnisse zu einen großen Konstrukt zusammen. Wie Richard Elliot Friedmann schon sagte: \glqq Wir zerlegen den Glauben in immer kleinere Bestandteile, aber wir vergessen, alles wieder zu einen großen Ganzen zusammenzufügen\grqq. Schlussendlich sollte Theologie sich nicht als eine Wissenschaft verstehen, sondern eine Lehre mit Wissenschaft als Instrument.
