% !TEX root = Lehren__von_Jesus_Christus.tex
\chapter{Appendix: Motive, Gruppen und Symbole in den Evangelien}
In diesen Kapitel sollen nochmal Motive, Gruppen und Symbole des neuen Testaments gesammelt kurz vorgestellt und erklärt, auch wenn diese immer wieder bei der entsprechenden Stellen bereits erklärt wurden.

\section{Religiöse Strömungen}
\subsection{Pharisäer}
Die Pharisäer erwarteten das Heil im Anblick des damaligen Leids nicht in der Gegenwart das Heilsgeschehen, sondern in der Zukunft. Wenn man sich um das überlieferte Gesetz Gottes bemühte und es peinlich genüg erfülle, dann werde Gott am Ende doch noch seinen Volk eine Zukunft eröffnen. Sie waren zum Großteil Bauern oder Handwerker, die eine offene und mitunter harte Streitkultur pflegten. Es ist anzunehmen, dass Jesus und seine Familie auf dieser Gruppierung zuzuordnen waren.

\subsection{Saddzuäer}
Die Sadduzäer war ihre politische Rolle öfters wichtiger als ihre Religiöse. Das daraus resultierende Anpassen an die römische Staatsmacht gab ihnen begrenzt politische Macht und sie stellten die Priester in Jersualem. Sie gehörten der sozialen und politischen Oberschicht an und versuchten, die blutigen Umstände zu mindern und Ordnung im Land herzustellen. Sie waren vermutlich die Hauptverantwortlichen hinter dem Prozess um Jesus und der Kerngedanke war, lieber einen Menschen hinzurichten als ``das ganze Volk verderben zu sehen''.

\subsection{Essener}
Sie zogen sich komplett aus dem öffentlichen Leben hinaus und ging in die Wüste, um nach den eigenen Gesetzen und Überzeugungen zu leben. Ob sie tatsächlich mit dem Wüstenkloster in Qumran in Verbindung zu bringen sind, erscheint inzwischen fraglich, jedoch ist es wahrscheinlich widerlegt, dass Jesus diesen Kreisen angehört habe.

\subsection{Zeloten}
Die Zeloten waren die damaligen Nationalhelden. Sie widersetzten sich der römischen Zwangsherrschaft mit einen rücksichtslosen Partisanenkrieg. Sie nahmen hierfür willentlich die Kreuzigung in Kauf und kämpften mit dem Wissen, eigentlich keine Chance zu haben.


\section{Hölle}
In der Bibel gibt es drei Wörter für die Hölle: Scheol, Hades und Gehenna. Scheol stammt aus den Alten Testament und meint so etwas wie das Totenreich.
Dieses Wort ist hierbei bei den eher poetisch einzuordnenden Texten vorzufinden. Hierbei wird das Wort oft als Synonym wie für Tod, Grab oder Grube verwendet.
Somit war das Scheol erstmal der Tod selbst und er herrschte kein Bild vor, in den die unsterbliche Seele in irgendeiner Weise weiterexisitert.
\\
Durch den Einzug des persischen Dualismus wurde neben den Engel, Teufel und Dämonen auch die apokalyptischen Lehren im Judentum geboren. In dieser
wird wie in Kapitel\ \ref{sec:Teufel} erläutert die Schuld des Bösen an die bösen Mächte abgeschoben und die Idee des jüngsten Gerichts eingeführt,
in der Gott die Gerechten belohnt und die Ungerechten bestraft. Für diesen Zweck werden die Körper der Toten wiedererweckt, also aus dem Scheol zurückgeholt.
Hierbei musste es als eine Erweckung des Körpers verstanden werden, da in jüdischer Tradition nur ein Körper leben kann. Für die gerechten
wird es eine gerechte Welt ohne Leid und Sorgen geben und für die Ungerechten eine Bestrafung. Ein dualistisches Denken, in dem es unterschiedliche Totesschicksale für Ungerechte und Gerechte gibt oder eine Erettung aus den Gottesreich möglich ist, gibt es selten.  Jedoch gehört hier auch das Totenreich in Gottes Machtbereich.
\\~\\
Im neuen Testament wird von der Hölle als Hades und von Gehenna gesprochen. Ersteres ist die griechische Übersetzung des Wortes Scheol und meint somit dasselbe Totenreich. Letzteres ist schlicht ein Tal neben Jerusalem, in dem offenbar in einen Kult Kinder und andere Sachen als Opfer verbrannt wurden und als Müllhalde mit vielen Toten Tieren, Kadavern etc, diente. An diesen Ort wurden die Leichen von Toten entsorgt, wenn diese nicht beerdigt werden konnten
oder durften. So beerdigt zu werden war für die damaligen Menschen eine furchterregende Vorstellung. Dieser Ort war somit quasi ein \glqq Hölle auf Erden\grqq{}, welcher sich jedoch auf Erden befindet und kein übernatürlicher Ort ist. Wenn Jesus von Gehenna und Hades spricht, so meinte er nach jüdischen Verständnis das jüngste Gericht, in den die Ungerechten zwar nicht ewig gefoltert werden, aber deren Existenz nach dem Gericht ausgelöscht wird.