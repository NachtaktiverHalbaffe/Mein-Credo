% !TEX root = Lehren__von_Jesus_Christus.tex
\chapter{Appendix: Motive, Gruppen und Symbole in den Evangelien}
In diesen Kapitel sollen nochmal Motive, Gruppen und Symbole des neuen Testaments gesammelt kurz vorgestellt und erklärt, auch wenn diese immer wieder bei der entsprechenden Stellen bereits erklärt wurden.

\section{Religiöse Strömungen}
\subsection{Pharisäer}
Die Pharisäer erwarteten das Heil im Anblick des damaligen Leids nicht in der Gegenwart das Heilsgeschehen, sondern in der Zukunft. Wenn man sich um das überlieferte Gesetz Gottes bemühte und es peinlich genüg erfülle, dann werde Gott am Ende doch noch seinen Volk eine Zukunft eröffnen. Sie waren zum Großteil Bauern oder Handwerker, die eine offene und mitunter harte Streitkultur pflegten. Es ist anzunehmen, dass Jesus und seine Familie auf dieser Gruppierung zuzuordnen waren.

\subsection{Saddzuäer}
Die Sadduzäer war ihre politische Rolle öfters wichtiger als ihre Religiöse. Das daraus resultierende Anpassen an die römische Staatsmacht gab ihnen begrenzt politische Macht und sie stellten die Priester in Jersualem. Sie gehörten der sozialen und politischen Oberschicht an und versuchten, die blutigen Umstände zu mindern und Ordnung im Land herzustellen. Sie waren vermutlich die Hauptverantwortlichen hinter dem Prozess um Jesus und der Kerngedanke war, lieber einen Menschen hinzurichten als ``das ganze Volk verderben zu sehen''.

\subsection{Essener}
Sie zogen sich komplett aus dem öffentlichen Leben hinaus und ging in die Wüste, um nach den eigenen Gesetzen und Überzeugungen zu leben. Ob sie tatsächlich mit dem Wüstenkloster in Qumran in Verbindung zu bringen sind, erscheint inzwischen fraglich, jedoch ist es wahrscheinlich widerlegt, dass Jesus diesen Kreisen angehört habe.

\subsection{Zeloten}
Die Zeloten waren die damaligen Nationalhelden. Sie widersetzten sich der römischen Zwangsherrschaft mit einen rücksichtslosen Partisanenkrieg. Sie nahmen hierfür willentlich die Kreuzigung in Kauf und kämpften mit dem Wissen, eigentlich keine Chance zu haben.