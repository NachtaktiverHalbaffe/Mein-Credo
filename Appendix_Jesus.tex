% !TEX root = Lehren__von_Jesus_Christus.tex
\chapter{Appendix: Motive, Gruppen und Symbole in den Evangelien}
In diesen Kapitel sollen nochmal Motive, Gruppen und Symbole des neuen Testaments gesammelt kurz vorgestellt und erklärt, auch wenn diese immer wieder bei der entsprechenden Stellen bereits erklärt wurden.

\section{Religiöse Strömungen}
\subsection{Pharisäer}
Die Pharisäer erwarteten das Heil im Anblick des damaligen Leids nicht in der Gegenwart das Heilsgeschehen, sondern in der Zukunft. Wenn man sich um das überlieferte Gesetz Gottes bemühte und es peinlich genüg erfülle, dann werde Gott am Ende doch noch seinen Volk eine Zukunft eröffnen. Sie waren zum Großteil Bauern oder Handwerker, die eine offene und mitunter harte Streitkultur pflegten. Es ist anzunehmen, dass Jesus und seine Familie auf dieser Gruppierung zuzuordnen waren.

\subsection{Saddzuäer}
Die Sadduzäer war ihre politische Rolle öfters wichtiger als ihre Religiöse. Das daraus resultierende Anpassen an die römische Staatsmacht gab ihnen begrenzt politische Macht und sie stellten die Priester in Jersualem. Sie gehörten der sozialen und politischen Oberschicht an und versuchten, die blutigen Umstände zu mindern und Ordnung im Land herzustellen. Sie waren vermutlich die Hauptverantwortlichen hinter dem Prozess um Jesus und der Kerngedanke war, lieber einen Menschen hinzurichten als ``das ganze Volk verderben zu sehen''.

\subsection{Essener}
Sie zogen sich komplett aus dem öffentlichen Leben hinaus und ging in die Wüste, um nach den eigenen Gesetzen und Überzeugungen zu leben. Ob sie tatsächlich mit dem Wüstenkloster in Qumran in Verbindung zu bringen sind, erscheint inzwischen fraglich, jedoch ist es wahrscheinlich widerlegt, dass Jesus diesen Kreisen angehört habe.

\subsection{Zeloten}
Die Zeloten waren die damaligen Nationalhelden. Sie widersetzten sich der römischen Zwangsherrschaft mit einen rücksichtslosen Partisanenkrieg. Sie nahmen hierfür willentlich die Kreuzigung in Kauf und kämpften mit dem Wissen, eigentlich keine Chance zu haben.

\section{Übernatürliche Wesen}
Neben Jesus und Gott kommen in der Bibel noch andere, übernatürliche Wesen vor. Diese sind im Wesentlichen Engel, der Teufel und Dämonen. Ziel ist die Falsifizierung oder Valdierung der Existenz dieser als eigenpersonelle Wesen und ggf.\ die Identifizierung für was sie stehen bzw.\ was sie repräsentieren.

\subsection{Engel}
Engel existierten bereits in verschiedenen, vorchristlichen Religionen. So stammen von dort z.B. auch die Bilder der Cherubinen und Seraphimen. Ebenso
stammen aus den Zeiten des persischen Dualismus z.B. auch Wesensbilder, aus denen später die Schutzengel hervorgehen werden. Spätestens zu Zeiten des Neuen Testamentes gehören Engel und Engel-ähnliche Wesen zum altorientalischen Weltbild, die in vielen Religionen und Gegenden bekannt waren. Daher ist anzunehmen, dass durch diesen Einfluss Engel Einzug in die Bibel gefunden haben. Ein Hinweis hierfür ist, dass Engel erst in den älteren Schriften des Alten Testamentes stärker in den Vordergrund gerückt sind, in denen das jüdische Volk bereits durch einige Exile und der damaligen Umwelt
unter erheblichen Einfluss im Bezug zu diesen Wesen stand.
\\~\\
In den Alten Testament werden Engelswesen verwendet, wenn Gott mit den Menschen gesprochen und interagiert hat, aber die Menschen (Ehr-) Furcht hatten
Gott direkt zu nennen. Im neuen Testament verstärkt sich dieses Bild und dort treten Engel als Boten und Künder von Gott auf.  Die Evangelien als auch somit
verbunden Jesus begeben sich hierbei in das damalige Weltbild herein und haben die Engel als Wesen stehen lassen, da diese kein Hindernis für die Verkündigung
der Inhalte waren. Sie haben aber keine eigene Engelslehre entwickelt und sind somit als reine stilistische Mittel der damaligen Zeit zu sehen, die Jesus und die Autoren benutzt haben. \\

Die ersten Kirchenväter ordnen den Engeln keine besondere Bedeutung zu. Für sie sind Engel dienende Gottes und es ist verboten, diese anzubeten. Auch wird einen Christen nicht vorgeschrieben, an eine vorgegebene Engelslehre zu glauben. Erst nach 500 findet über die Kirchengeschichte ein überhobenes Engelsbild Einzug, doch diese sind nicht eindeutig biblisch und lassen sich in Zeiten einordnen, in denen einige Sachen in der Kirche schief lief
 (Kreuzzüge, Heiligen- und Marienverehrung, Ablaßkulte, Höllenlehre etc.).
\\~\\
Demnach können die Engel auf den Einfluss des damaligen Weltbildes und den darin vorherrschenden Religionen und Kulturen zurückgeführt werden. Da die Engel nur ein verlängerter Arm Gottes waren und keine Eigenverhalten aufgewiesen haben, können Engel als Metaphern für die Eigenschaften und Verhaltensweisen Gottes verstanden und somit aus den Geschichten gestrichen werden, ohne deren Sinn und Aussagen zu verändern. Da Gott nur in Metaphern
von Menschen beschrieben und verstanden werden kann, ist das Vorkommen von Engeln in der Bibel als metaphorische Bild auch nichts schlimmes, sondern sogar etwas förderliches, solange man dieses Bild immer wieder auf Gottes Wesen zurückführt und Engel nicht als eigenpersonelle Wesen überhöht.

\subsection{Teufel}\label{sec:Teufel}
Auf die Vorstellung eines Gegenspielers in Form eines Teufels geht aus den Parsismus (persischer Dualismus) hervor. Etwa im 2. Jahrhundert vor Christus findet daher das Bild des Teufels Einzug in das Glaubensgut der damaligen Welt. So ist der Teufel der Schuldträger für das ganze Böse in der Welt. Daher ist die Entstehung
des Teufel auch auf den Einfluss der damaligen, persischen Kultur und Glaubens zurückzuführen. Dies wird dadurch untermauert, dass der Teufel in seiner Hintergrundgeschichte weiter in außer--biblischen Schriften, meist in Volkssagen, weiter charakterisiert und dann in der Bibel als gekanntes Wissen vorausgesetzt werden.
\\~\\
Im Alten Testament ist der Satan kein Gegenspieler, sondern ein Diener Gottes, der in dessen Auftrag handelt. Durch den zuvor genannten Einfluss kam dann auch das Bild des Teufels in die heiligen Schriften. Hierbei wurde aufgrund eines Übersetzungsfehlers der zuvor genannte Satan mit den Teufel gleichgesetzt.
Daher muss der Satan im Alten Testament und der Teufel im Neuen Testament getrennt betrachtet werden.
\\~\\
Im Alten Testament kommt bis auf die bekannte Hiobsgeschichte, die nicht in den heiligen Schriften des Judentums vertreten ist, kaum eine Erwähnung des Satans vor. Daher spielt der Satan im Alten Testament eine untergeordnete Rolle und kann als Metapher für den prüfenden Gott gesehen werden. \\ Im Neuen Testament ist das Glauben an der Existenz an Engel, Teufel und Dämonen bereits in der Bevölkerung etabliert. Wie bei den Engeln ist anzunehmen, dass Jesus nicht an die Existenz des Teufels glaubt, sich aber dieser stilistisch bedient, da es seiner eigentlichen Botschaft nicht im Weg steht. Im Neuen Testament steht der Teufel als Metapher
für die Bereitschaft  des Menschen zum Bösen.
\\~\\
Bei den Kirchenvätern wird unterschiedlich von den Teufel gesprochen. Die Tendenz ist aber eher, diesen als fiktive Figur zu betrachten. Außerdem wird dem Teufel nie das böse zugeschrieben, sondern nur die Rolle des Versuchsers und Verführers zu den Bösen. Das Böse geht auch bei dem Glauben an einen Teufel immer vom Menschen selbst aus. Ähnlich wie bei der Einführung des Teufels ist der Glauben an die Existenz des Teufels bei den Kirchenvätern immer auf den kulturellen und spirituellen Einfluss derjenigen Zeit zurückzuführen.
\\~\\
Zuletzt ergibt die Existenz des Teufels und seine Hintergrundgeschichte keinen Sinn. Dass sich vollkommene Wesen gegen Gott in einen Kampf einlassen, in den sie nur verlieren können, macht keinen Sinn. Zudem ist keine Motivation für den Abstieg eines vollkommenen zu einen unvollkommenen Wesens nicht gegeben.
Außerderm taugt die Figur des Teufels nicht für die Deutung der Welt und löst keine Probleme.
\\~\\
Daher ist eine Existenz des Satans und Teufels als eigenpersonelle Wesen abzulehnen und auch nicht förderlich für einen gesunden Glauben. Außerdem eröffnet eine Teufelsfigur den Menschen die lukrative und schädliche Möglichkeit, den Schuld und das Böse weg von sich selber und von Gott hin zu einen Sündenbock wegzuschieben.

\subsection{Dämonen}
Die Dämonen sind im Judentum Geistwesen, die rein negativ eingeschätzt werden und für Krankheiten und Tod verantwortlich sind. Diese haben ebenfalls über den Dualismus Einzug in das Glaubensgut des Neuen Testament gefunden. Daher sind diese ebenfalls nach den selben Schema wie bei den Teufel und Engeln auf Einflüsse von anderen Religionen und Kulturen zurückzuführen. Daher wird für genauere Infos auf\ \ref{sec:Teufel} verwiesen. Im Laufe der Kirchengeschichte entwickelten sich dann Dämonen auch zu Anstiftern und Verführer des Bösen weiter.
\\~\\
Es gelten die gleichen Schlussfolgerungen wie in Kapitel\ \ref{sec:Teufel}, außer dass hier Dämonen für keine Metapher stehen, sondern schlichtweg nicht existieren und dementsprechend als schlichte Krankheitserscheinungen gesehen werden können.

\section{Hölle}
In der Bibel gibt es drei Wörter für die Hölle: Scheol, Hades und Gehenna. Scheol stammt aus den Alten Testament und meint so etwas wie das Totenreich.
Dieses Wort ist hierbei bei den eher poetisch einzuordnenden Texten vorzufinden. Hierbei wird das Wort oft als Synonym wie für Tod, Grab oder Grube verwendet.
Somit war das Scheol erstmal der Tod selbst und er herrschte kein Bild vor, in den die unsterbliche Seele in irgendeiner Weise weiterexisitert.
\\
Durch den Einzug des persischen Dualismus wurde neben den Engel, Teufel und Dämonen auch die apokalyptischen Lehren im Judentum geboren. In dieser
wird wie in Kapitel\ \ref{sec:Teufel} erläutert die Schuld des Bösen an die bösen Mächte abgeschoben und die Idee des jüngsten Gerichts eingeführt,
in der Gott die Gerechten belohnt und die Ungerechten bestraft. Für diesen Zweck werden die Körper der Toten wiedererweckt, also aus dem Scheol zurückgeholt.
Hierbei musste es als eine Erweckung des Körpers verstanden werden, da in jüdischer Tradition nur ein Körper leben kann. Für die gerechten
wird es eine gerechte Welt ohne Leid und Sorgen geben und für die Ungerechten eine Bestrafung. Ein dualistisches Denken, in dem es unterschiedliche Totesschicksale für Ungerechte und Gerechte gibt oder eine Erettung aus den Gottesreich möglich ist, gibt es selten.  Jedoch gehört hier auch das Totenreich in Gottes Machtbereich.
\\~\\
Im neuen Testament wird von der Hölle als Hades und von Gehenna gesprochen. Ersteres ist die griechische Übersetzung des Wortes Scheol und meint somit dasselbe Totenreich. Letzteres ist schlicht ein Tal neben Jerusalem, in dem offenbar in einen Kult Kinder und andere Sachen als Opfer verbrannt wurden und als Müllhalde mit vielen Toten Tieren, Kadavern etc, diente. An diesen Ort wurden die Leichen von Toten entsorgt, wenn diese nicht beerdigt werden konnten
oder durften. So beerdigt zu werden war für die damaligen Menschen eine furchterregende Vorstellung. Dieser Ort war somit quasi ein \glqq Hölle auf Erden\grqq{}, welcher sich jedoch auf Erden befindet und kein übernatürlicher Ort ist. Wenn Jesus von Gehenna und Hades spricht, so meinte er nach jüdischen Verständnis das jüngste Gericht, in den die Ungerechten zwar nicht ewig gefoltert werden, aber deren Existenz nach dem Gericht ausgelöscht wird.