% !TEX root = Theologische_Grundfragen.tex
\chapter{Die Verteididung der Willensfreiheit des Menschen} \label{Willensfreiheit}
Die Willensfreiheit des Menschen ist in dieser vertretenen Theologie einer der zentralen Grundpfeiler. Ohne diese kann die Frage nach dem Leid nicht ansatzweise zufriedenstellend beantwortet werden (siehe Kapitel \ref{Theodizee}). Daher wird die Willensfreiheit des Menschen hier in einen extra Kapitel in allen seinen Facetten genauer betrachtet. Das Ziel dieses Kapitels ist daher die Verteididung der Willensfreiheit des Menschen.\\

\section{Allgemeine Verteidigung der Willensfreiheit des Menschen}
Die Freiheit des Menschen bildet die grundlegende Voraussetzung für moralisches Handeln. Moral setzt notwendig voraus, dass der Handelnde zwischen verschiedenen Handlungsoptionen wählen kann. Nur wenn ein Mensch die Möglichkeit hat, anders zu handeln, kann man ihm Verantwortung zuschreiben. Ohne Freiheit wären Lob und Tadel, Schuld und Verdienst bedeutungslos, da alle Handlungen lediglich kausal determiniert wären. Freiheit ist daher nicht nur ein metaphysisches Postulat, sondern ein praktischer Grundbegriff der Ethik.\\

Diese Freiheit lässt sich jedoch nicht durch theoretische Vernunft nachweisen. Der Versuch, Freiheit auf logischem oder metaphysischem Wege zu beweisen, stößt an Grenzen, da sich der Mensch als Teil der Natur auch den Gesetzmäßigkeiten von Ursache und Wirkung unterworfen sieht. Ebenso wenig kann die empirische Wissenschaft Freiheit widerlegen oder bestätigen: Sie untersucht Handlungen und Entscheidungen zwar auf ihre neuronalen und psychologischen Bedingungen hin, doch diese erklären lediglich die Bedingungen des Handelns, nicht die Möglichkeit freier Wahl selbst. Empirische Daten erfassen Ursachen, aber nicht die normative Dimension, die moralische Verantwortung erst ermöglicht.\\

Stattdessen zeigt sich Freiheit im praktischen Vollzug. In moralischen Entscheidungssituationen erlebt sich der Mensch als frei und verantwortlich. Dieses praktische Bewusstsein der Freiheit ist nicht Ergebnis einer theoretischen Ableitung, sondern Bedingung der Möglichkeit moralischen Urteilens überhaupt. Schon Kant hat darauf hingewiesen, dass Freiheit als Postulat der praktischen Vernunft unverzichtbar ist: Damit moralische Gesetze sinnvoll sein können, muss der Mensch als frei gedacht werden.\\

Somit erweist sich die Freiheit des Menschen nicht als empirisch beweisbare Tatsache, sondern als notwendige Voraussetzung des moralischen Handelns. Weder Theorie noch Wissenschaft können sie aufheben, ohne zugleich die Grundlage ethischer Praxis zu untergraben. Freiheit bleibt damit ein unverzichtbares Postulat, das den Raum eröffnet, in dem moralisches Handeln und Verantwortung überhaupt erst denkbar werden.

\section{Verteidung, dass Jesus die Willensfreiheit des Menschen nicht negiert}
Es könnte argumentiert werden, dass Jesus Wirken die Willensfreiheit der Menschen eingeschränkt hat, da durch seine übernatürliche Fähigkeiten, welche ihn durch seine Gottessohnschaft verliehen wurden, Menschen zu bestimmten Handlungen gezwungen oder beeinflusst hat. Wer kann schon noch frei an Gott glauben, wenn Gott selbst Mensch geworden und in dieser Erde missoniert hat. Dies könnte als Widerspruch zur Vorstellung der Willensfreiheit interpretiert werden. Daher soll dies hier verteidigt werden.\\

Zuallerst muss festgehalten werden, dass Jesus seine Fähigkeiten nie missbraucht hat. Jesus hat niemanden gezwungen, ihm zu folgen oder an ihn zu glauben. Vielmehr hat er den Menschen die Möglichkeit gegeben, sich frei für oder gegen ihn zu entscheiden. Dies war sogar ein Kernelement seine Versuchungen durch den Teufel, welchen er widerstanden hat: Benutze deine Macht, um die Menschen dadruch zu einen Glauben an dich zu zwingen (siehe Kapitel \ref{Versuchung}).\\

Zudem durchzieht sich Jesus handeln immer mit offenen Möglichkeiten, mit welcher seine Gottessohnschaft und Handeln angezweifelt oder ignoriert werden kann. Dies beginnt schon bei seiner Geburt: Die Umstände seiner Herkunft sind unsicher. Niemand weiß, ob Maria wirklich durch den Heiligen Geist schwanger wurde oder ob Josef der leibliche Vater von Jesus war. Er wurde auch nicht in ein mächtiges Haus hineingeboren, welches man nicht anzweifeln konnte. Stattdessen waren die einzigen Zeugen diejenigen, welche von der Gesellschaft eh schon ausgestoßen und damit deren Wahrheitsgehalt grundsätzlich angezweifelt wurde. \\

Auch seine Wunder wurden oft angezweifelt und als Zauberei oder Betrug abgetan. Noch bis heute wird sich gestritten, auf welche Art und Weise die Überlieferungen über seine Wunder interpretiert werden müssen. Selbst bei seiner Auferstehung ist er nicht direkt vom Kreuz gestiegen. Nein, er ist in der geschlossenen Grabkammer auferstanden, fernab von allen Zeugen außer der handverlesenen Auswahl von bereits Gläubigen Menschen. Dies bietet heute auch noch viel Angriffsfläche für den christlichen Glauben.\\

Langsam wird klar, dass gerade die Punkte, welche seine Legitimität als Gottes Sohn und Messias untermauern sollten, immer wieder angezweifelt werden konnten. Dies bietet die Möglichkeit, diese Angriffsfächen als einen bewussten Teil seines Wirkens zu interpretieren, mithilfe dessen die Freiheit des Menschen gewährleistet werden konnte. Jesus wollte die Menschen nicht zwingen, an ihn zu glauben. Vielmehr wollte er ihnen die Möglichkeit geben, sich frei für oder gegen ihn zu entscheiden. Dies ist ein zentraler Bestandteil seiner Botschaft der Liebe und Freiheit. Und somit kann zurecht behauptet werden, dass seine Existenz und Wirken genau so gestaltet war, sodass die Freiheit des Menschen gewahrt bleiben konnte.\\

\section{Verteididung, dass die Hirnforschung die Willensfreiheit des Menschen nicht negiert}
In den letzten Jahrzehnten haben Fortschritte in der Neurowissenschaft zu Diskussionen über die Willensfreiheit geführt. Einige Studien, wie die von Benjamin Libet in den 1980er Jahren, deuten darauf hin, dass Entscheidungen im Gehirn getroffen werden, bevor wir uns ihrer bewusst werden. Dies hat zu der Annahme geführt, dass unser Gefühl der freien Willensentscheidung eine Illusion sein könnte.\\
Hier ist es Libet selbst, der betont, dass seine Ergebnisse nicht notwendigerweise die Existenz des freien Willens widerlegen. Er schlägt vor, dass es eine Art \glqq Veto-Macht\grqq{} geben könnte, bei der das Bewusstsein die Möglichkeit hat, eine bereits initiierte Handlung zu stoppen. Dies deutet darauf hin, dass das Bewusstsein immer noch eine Rolle im Entscheidungsprozess spielen könnte, auch wenn es nicht der ursprüngliche Auslöser ist. Dies ermöglicht wieder einen fundierten Glauben an die Willensfreiheit des Menschen.\\
Zudem ist ebenfalls nicht klar, ob rein motorische Entscheidungen, wie die in Libets Experimenten, auf komplexe moralische Entscheidungen übertragbar sind. Moralische Entscheidungen beinhalten oft Überlegungen, Werte und soziale Kontexte, die weit über einfache motorische Handlungen hinausgehen. Es ist möglich, dass das Bewusstsein in diesen komplexeren Situationen eine größere Rolle spielt. Oder dass wir sehr wohl einen freien Willen haben, aber für die alltäglichen, automatisierten Aufgaben sich unser Wille entscheidet alles komplett an unser Unterbewusstsein abzugeben.\\
Zuletzt konnten neuere Studien die Ergebnisse von Libet teilweise oder ganz widerlegen. So wurde die zuvor genannte Veto-Macht in einigen Experimenten bestätigt, was darauf hindeutet, dass das Bewusstsein tatsächlich eine Rolle im Entscheidungsprozess spielen könnte. Andere Studien haben gezeigt, dass die zeitliche Abfolge von Gehirnaktivität und bewusster Entscheidung komplexer ist als ursprünglich angenommen. Zudem zeigte eine Forschung, dass dieses Bereitschaftspotential mit anderen Dingen wie Atmung zusammenhängen.\\

Insgesamt bleibt die Frage der Willensfreiheit trotz der Herausforderungen durch die Neurowissenschaften offen. Während einige Studien darauf hindeuten, dass unser Gefühl der freien Willensentscheidung komplexer ist als ursprünglich angenommen, gibt es keine endgültigen Beweise dafür, dass der freie Wille eine Illusion ist (und fairerweise auch für die gegenteilige Hypothese). Vielmehr deuten viele Erkenntnisse darauf hin, dass das Bewusstsein und das Gehirn auf komplexe Weise zusammenarbeiten, um Entscheidungen zu treffen. Daher bleibt ein fundierter Glaube an die Willensfreiheit des Menschen weiterhin gerechtfertigt.