% !TEX root = Lehren__von_Jesus_Christus.tex
\chapter{Sakramente und Rituale}\label{sec:Sakramente}

\section{Die Grundproblematik des Begriff Sakraments}
Das Wort Sakrament is ein nicht-biblischer Begriff, spielt jedoch eine zentrale Rolle über alle Konfessionen hinweg.
Nach den Augsburger Bekenntnis sind Sakramente direkt eingesetzt durch Jesus Christus und haben ein materielles Zeichen.
Nach dieser Definition ist aber z.B. die Taufe kein Sakrament. Luther sprich daher auch die Problematik dieses Begriffs an.
Für ihn gibt es nur ein Sakrament: Die Sündenvergebung. Die sakramentalen Zeichen sind Konkretisierungen dieses Grundsakraments.
Daraus geht klar hervor, dass die Kirche das Sakrament zwar feiern und vollziehen kann, es jedoch nicht verleihen kann, denn
Sündenvergebung geht direkt von Gott aus. Calvin erweitert diese Definition passend und soll aus Ausgangssituation für die Betrachtung
des Themas dienen:
\begin{verse}
    Ein Sakrament ist ein Zeugnis der göttlichen Gnade, das durch ein öußerliches Zeichen bestätigt ist.\ mit einer entsprechenden Bezeugung
    unserer Frömmigkeit ihn gegenüber.
\end{verse}
Trotz der nicht-biblisch Herkunft und der Definitionsproblematik sind Sakramente und Rituale wichtig und essentiell.
Selbst Jesus Christus hat uns in Form des Abendmahls Rituale gutgeheißen und weitergegeben. Es gilt daher, Sakramente und weiterführend
Rituale näher zu betrachten und eindeutige Kriterien für gesunde und legitimierbare Rituale zu finden. Gemeinschaften brauchen am Ende
verlässliche und etablierte Handlungen, die von den Einzelnen in der Gemeinschaft geteilt werden. Am Ende gilt dies auch für die Glaubensgemeinschaft.
Daher sollen weitergehend Kriterien für gute Sakramente und Rituale herausgearbeitet werden.

\section{Menschengemachte Rituale und Sakramente}
\section{Nicht alle Sakramente und Rituale sind heilig und Bedingung für das Heil}
\section{Es gibt meistens nicht die Wahrheit}
\section{Im Wandel der Zeit}