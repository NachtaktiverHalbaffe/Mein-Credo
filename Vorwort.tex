% !TEX root = Mein Credo.tex
\part{Vorwort}
\chapter*{Vorwort}
Seit meiner Geburt wuchs ich nach protestantischen Glauben auf, nahm dessen Werte an, sog dessen Inhalt auf und spätestens ab meinen 13. Lebensjahr war mein Glauben ein essentieller Bestandteil meines Lebens.
\\~\\
Wie jeder Mensch entwickelte ich mich. Mein Leben war eine Achterbahnfahrt. Als Kind sog ich die Inhalte des Glaubens in mich auf. Es war nur ein oberflächliches ankratzen der wirklich wichtigen Sachverhalte, doch ich war begeistert. Da ich nicht getauft war, hab ich darauf bestanden, getauft zu werden. Nach langen bearbeiten meiner Mutter wurde ich am 1. Juli 2007 getauft. Ich dachte, dass nichts und niemand in meinen Leben meinen glauben im Weg stehen könnte. Doch es kam eine  Zeit der persönlichen Herausforderungen. Im Angesicht diesen Herausforderungen konnte ich immer weniger an Gott glauben. Auch mithilfe meines Glaubens bezwang ich jedoch diese Situationen. Der Glauben prägte mein Leben und ließ gewisse Werte in mich einbrennen.
\\~\\
Über die Jahre zweifelte ich wie jeder gute Christ an meinen Glauben und begann die Sachverhalte zu hinterfragen. Da ich auch (natur-) wissenschaftlich veranlagt bin  -wahrscheinlich habe ich das durch meinen Vater geerbt- wollte ich meinen Glauben wissenschaftlich und theologisch hinterfragen. In meinen 17. Lebensjahr entschied ich mich, meinen Glauben von Grund auf kritisch zu hinterfragen und daraus mein Credo zu ziehen. Mit dem Gedanken im Hinterkopf, dass dies hier subjektiv geprägt sein wird und es kein absolutes, rationales Ergebnis bei einen irrationalen Thema möglich ist, beginne ich, meine komplizierten Denkvorgänge, Folgerungen, Schlüsse, Ansätze u.v.m.\ u.v.m. hier zu verschriftlichen mit dem Ziel, meinen eigenen Credo aufzustellen und hier festzuhalten, woran ich Glaube.
\\~\\
Egal ob mein Credo zu einen zufriedenstellenden Ergebnis kommt oder nicht, egal ob ich in ferner Zukunft immer noch der Kirche angehöre oder nicht, ich bereue meinen Glauben nicht. Ich bin mir sicher, dass die Erfahrungen und Werte, die mich geprägt haben, in jeder Weise im Leben weiterhelfen werden. Als Dokumentation halte ich hier nun alles für mich und meine Nachwelt schriftlich fest. Entschuldigen sie meine eher merkwürdige Sprache und meine für sie Wirr erscheinenden Gedankengänge. Dieses riesige Projekt wird mich hoffentlich eine sehr lange Zeit begleiten und hoffentlich zu einen Ende kommen. Wenn nicht, dann ist es so. Wenn ja, dann spiegelt dieses Stück nicht nur meinen Credo wieder, sondern ein Stück meines Lebens.
\\~\\
Nun stehe ich hier, bereit anzufangen. Vielleicht bin ich einfach nur verrückt. Vielleicht werde ich in ein paar Jahren nur schmunzelnd dies hier lesen und mich über meine eigene Naivität auslachen. Aber es ist mir egal. Ich habe mich immer wie ein verrückter Volldepp aufgeführt und es hat mich nie im Ansatz gejuckt.  Also starte ich dieses Projekt ohne den Gedanken, was andere hierüber denken können oder kurz gesagt: Es juckt mich nicht, was andere hierüber denken.
\newpage
Zum Projekt: Dies ist eine sehr, sehr stark zusammengefasste Version meines Wissens. Für mich wichtige Stellen werden ausführlicher erläutert und eher unbedeutende oder leicht zu erklärende Stellen stark komprimiert. Die hierfür herangezogenen Quellen werden kritisch hinterfragt und zusammengetragen, jedoch nicht wissenschaftlich korrekt zitiert. Einige Stellen werden eine einzige Redaktion vieler Quellen sein und ich werde einen Teufel tun, jede einzelne Quelle an den entsprechenden Stellen zu zitieren, da der Aufwand dafür zu groß wäre. Diese Arbeit dient ganz und allein mir und Gott und wir pfeifen auf banale Formalitäten. Die Quellen werden am Ende aufgeführt und es steht jeden frei, die Quellen zu überprüfen. Dies ist mein Credo und keine wissenschaftliche Thesis. Formalitäten spielen hier eine untergeordnete Rolle. Umso mehr spielt der Inhalt eine Rolle. Und der ließt sich besser ohne 3 Milliarden Quellenverweise.
\bigskip
\begin{flushright}
\textit{Aichwald, den \today}
\end{flushright}
\bigskip
\bigskip
\bigskip
\bigskip
\bigskip
\bigskip
\bigskip
\bigskip
\bigskip
\begin{figure}[h]
\begin{flushright}
\includegraphics[scale=0.7]{Signatur.jpg}\\
$\overline{~~~~~~~~~\mbox{Felix Brugger}~~~~~~~~~}$
\end{flushright}
\end{figure}