% !TEX root = Lehren__von_Jesus_Christus.tex
\chapter{Taufe}
Auf die Geburt und die Jugendgeschichte von Jesus folgt in allen 4 Evangelien die Vorstellung von Johannes den Täufer und dessen Taufrituals, welches sich Jesus anschließend selbst unterwirft. Daher soll als nächstes das Thema Taufe näher betrachtet werden.

\section{Kontextualisierung}
Im alten Testament und somit im alten Judentum spielte das Konzept der Reinheit eine große Rolle. Nach damaligen Verständnis war Reinheit das Gott gemäße und wiederum (kulturelle) Unreinheit ein Zustand, in dem man sich nicht dem Heiligen (z.B. Tempel, Synagogen usw.) nähern durfte. Die Begründung der Reinheitsgesetze waren durchaus komplex, aber nicht komplett theologischer Natur.
\\~\\
Im Hinblick zu der Taufe sind die damit verbundenen Reinigungsriten interessant, durch die bei Unreinheit der Zustand der Reinheit wiederhergestellt werden konnte. So gab es z.B. Mikwen als Ort der regelmäßigen Reinigung. Vermutlich nach der Zerstörung des Tempels wurde sogar ein Mikwenbad nach einer Konversion vorgeschrieben. Es gab aber auch andere Formen der Reinigung, die entweder selbstständig oder durch einen Dritten durchgeführt werden konnten. Am Ende musste die Reinigung von einen Priester bestätigt werden.
\\~\\
Erste Motive der Taufe gab es schon im Alten Testament. Dort wurde Naaman in einen Fluss gewaschen und somit bekehrt. Auch gaben Propheten o.ä.\ schon Konzepte in den Kontext von Moral und Ethik. Auch kündigten Propheten an, dass Gott selbst kommen und die Reinheit wiederherstellen wird. Auch wird Johannes der Täufer als Personifizierung von alttestamentlichen Motiven von Reinheit gesehen.
\\~\\
Wie zu sehen ist, gibt es schon im Alten Testament und somit im Judentum Motive und Riten zur Reinheit, die einen Hintergrund für die christliche Taufe bilden.

\section{Biblische Zeugnisse}
\subsection{Taufe des Johannes}
Über die Taufe des Johannes wird im  Neuen Testament berichtet, beim Markusevangelium ist sie sogar erste Bericht und somit der Anfang der Geschichte nach Markus Vorstellung. Bei Johannes kann die Taufe als die Reinigung der Sünden zum Zeitpunkt der Taufe als auch eschatologisch als Bewahrung vor der bevorstehenden Feuertaufe verstanden werden, wobei in der Urgemeinde die Endzeit schon als sehr nah erwartet wurde. Durch das Heilsgeschehen Jesu am Kreuz ist es ein gängiges theoligsches Bild geworden, dass sich in Jesu kommen sich die Tauferwartung Johannes erfüllt hat und die Partizipation an Jesus Heilsgeschehen nun als Tauferwartung in den Vordergrund gerückt ist. Bei der Taufe des Johannes gab es viele Elemente, von denen einige Bedeutung gewannen und andere an Bedeutung verloren:
\begin{itemize}
	\item Die Taufe geschieht im Jordan, einen Fluss. Im Johannesevangelium wird aber auch z.B. Ainon genannt als Wirkungsort des Johannes, weshalb der Jordan als Ort selber keine große Bedeutung haben wird für das Ritual.
	\item Die Taufe geschieht immer durch einen Täufer, der eine mediatorische Rolle einnimmt: Die Verkündigung und das Ritual selbst gibt nur Gottes Wirken weiter.
	\item Die Taufe war einmalig. Begründet wird dies durch die damalige eschatologische Naherwartung und der geforderten ethischen Neuorientierung, die nur einmalig sein konnte.
	\item Die Taufe hat eine Lebensführung nach der Thora als Konsequenz. Dabei ist die neue Lebensführung als natürlich kommende Konsequenz und nicht als Zwang interpretierbar. Wenn sich der Täufling der Taufe und somit Gott unterwirft, wird er sein Leben nach Gott aus freien Stücken ausrichten. Die Lebensführung nach der Thora kann in heutiger Zeit nicht als eine Lebensführung nach den Alten Testament, sondern nach Jesus Lehren verstanden werden, da er mit Gottes Autorität die Thora Neuinterpretiert und die Lehren reformiert hat.
	\item Die Taufe war kein Inititationsritus. Dennoch war sie ein Unterscheidungsmerkmal, was getaufte und die restliche jüdische Gemeinschaft unterschied.
\end{itemize}

In den Berichten des Matthäusevangelium wehrt Johannes das Taufbegehren Jesu ab: ``Ich habe nötig, von dir getauft zu werden'' (Mt 3,14). Somit wird die Taufe durch Jesus über die Wassertaufe als Ritual gestellt. Im Allgemeinen Fall macht dies kein Unterschied, da der Täufer selbst das Wirken Gottes und somit Jesus Wirken weitergibt, aber für einige Streitfragen wie z.B. die Behandlung von Ungetauften Toten oder ob für Ungetaufte auch die Erlösung Gottes gilt, gibt dies neuen Interpretationsspielraum.
\\~\\
In den Evangelien und zu der Wirkzeit Jesu wird von der Taufe primär im Zusammenhang mit der Johannestaufe berichtet. Auch gibt es Anspielungen und Umstände aus den Berichten, die eine direkte Tauftätigkeit Jesu oder eine indirekte durch seine Jünger vermuten lässt. Somit ist anzunehmen, dass die Taufe als Ritual von Jesus selbst praktiziert und somit akzeptiert wurde, jedoch stellt Jesus nirgendwo die Taufe als Bedingung für z.B. die Heilserlösung dar. Es ist also unbestreitbar, dass die Taufe als kirchliches und biblisch fundamentiertes Ritual/Sakrament interpretiert werden kann, die Verknüpfung der Taufe als fundamentale Bedingung für z.B. Aufnahme in die Gemeinschaft oder Zuspruch von Gottes Erlösung usw.\ kann jedoch nicht trivial beantwortet werden.

\subsection{Übergang der Johannestaufe zur Taufe der Urgemeinde}
Einerseits kann man annehmen, dass die Taufe Jesu als Vorbild übernommen wurde, andererseits -und viel wahrscheinlicher- dass die Jesusanhänger nach Ostern das Bedürfnis hatten, die Zugehörigkeit zum Auferstandenen durch ein von ihnen bekanntes Ritual zu erleben. Dabei war Paulus einer der ersten Theologen, die die Taufe weiter auslegten und als Ritual verbreiteten.
\\~\\
Anmerkung (für alles was Paulus Lehren betrifft): Wichtig zu verstehen ist jedoch, dass Paulus im Grunde auch nur ein Theologe war, der Jesu Worte und Lehren auslegen wollte. Seine Lehren haben zwar durch seine Gotteserfahrung und den Grund, dass seine Briefe im Kanon sind eine erhöhte Aufmerksamkeit, jedoch hat seine Auslegung keineswegs einen Absolutheitsanspruch, diese haben nur Lehren, die auf Jesus bzw. Gott selbst zurückzuführen sind. Dies soll nicht alle Aussagen Paulus infrage stellen, aber seine Lehren dürfen kritisiert und als nicht absolut angenommen werden.
\\~\\
Paulus bzw.\ die Berichte über seine Taufen in der Apostelgeschichte sind die ersten Überlieferung mit einer Taufe auf den Namen Jesu bzw.\ später auf den trinitarischen Namen. Auch tritt mit Paulus die Bedeutung des Taufenden in den Hintergrund, da er im Grunde eh nur ein Vermittler ist und am Ende der Heilige Geist und der Täufling im Vordergrund stehen. Zu der Verfassung des Römerbriefes war die Taufe bekannt, auch wenn sich die Taufrituale untereinander noch unterschieden. Es stand also außer Frage, ob man sich Taufen lassen sollte.
\\~\\
Paulus sieht Taufe und den heiligen Geist sehr eng, jedoch verbindet er damit nicht eine Geistverleihung. Vielmehr verleitet das Wirken des heiligen Geistes den Täufling dazu, seine Schuld abwaschen zu lassen und so vor Gott als gerecht bestehen zu können, oder um es anders zu sagen:
\\
Das Wirken des heiligen Geistes bewegt den Täufling erst zur Taufe und somit zum Leben mit Gott.

\subsection{Deutung der Taufe außerhalb von den echten Paulusbriefen}
Im Kolloserbrief wird die Taufe mit einer Beschneidung aus den alten Judentum verglichen. Der Epheserbrief beschreibt die Taufe als Versiegelung, bei den der Heilige Geist das Siegel ist und der Versiegelte Gott gehört. Die Theologie der Versiegelung fruchtete aber erst richtig im 2 Jhdt.\ und noch nicht (nachgewiesen) in der Urgemeinde. Der Epheserbrief sieht die Taufe ebenfalls als ein Symbol der Einheit, da es ein Ritual ist was alle Christen teilen und verbindet. Der Hebräerbrief versteht die Taufe als Waschung und in Korinth gab es mehrere Taufen, wenn auch offen ist welche Form oder ob es sich z.B. auch um Waschungen handeln konnte. Wie zu sehen ist waren die Formen und Deutungen von Taufe in den Urgemeinden vielfältig.

\subsection{Lukanisches Doppelwerk}
Lukas hat mit Paulus die Gabe des heiligen Geistes im Verbund mit Taufe gemeinsam, jedoch unterscheiden sich die Wirkung, Reihenfolge und Art der Taufe. In der lukanischen Tauftheologie ist die Taufe zugleich Ausdruck der Umkehr und Gottes Gabe, die die Kirche vermittelt, aber nicht aktiv verleiht (da dies immer noch das Wirken des Heiligen Geistes ist). Die Taufe geschieht wie bei Paulus auf den Namen Christi und bewirkt Sündenvergebung \& Heil. Der Täufling empfängt den Geist (was ehrlicherweise aber auch außerhalb der Taufe geschehen kann) und die Taufe wird Verstanden als Initiation und Selbstverpflichtung.

\subsection{Taufbefehl im Matthäusevangelium}
Auch wenn der Taufbefehl nicht historisch sein wird, steckt dort drin kombiniert der Konsens der frühchristlichen Gemeinschaft: Der 1. Schritt zur Gewinnung der Jüngerschaft ist die Taufe und die Zugehörigkeit der Kirche ist die wirkungsvollste Wirkung der Taufe. Dieses Schemata ist in den vielen Bekehrungsgeschichten zu sehen, wo an erster Stelle nach der Entscheidung des Täuflings zu einen Leben mit Gott die Taufe war.

\subsection{Zwischenzusammenfassung der biblischen Zeugnisse}
Die Taufe des Johannes führte dazu, dass es sich als frühchristlicher Ritus herauskristallisierte. Im Neuen Testament entwickelte es sich zu einen selbstverständlichen Element der christlichen Identität. Alle Quellen hatten das Element des Wassers sowie das Zusammenwirken von Täufer und Täuflings gemeinsam. Die Taufe selber ist Gottes Wirken und gleichzeitig die Folge dessen, braucht aber einen Täufer, der eine Vermittlerrolle übernimmt. Das Heil kommt nicht von den Ritus selber, sondern geht am Ende immer von Gott aus. Wesentliche Elemente sind die Sündenvergebung, Reinigung und die Verbindung mit dem Wirken bzw.\ die Gabe des Geistes. Taufe ist mit den Geist verbunden, ohne die Geistbegabung an die Taufe zu binden. Ebenso verbindet die Taufe mit den Heilsereignis, ohne Heil an die Taufe zu binden. Die Taufe schafft Gemeinschaft und ist ein öffentliches Bekenntnis des Täuflings. Zuletzt stellt die Taufe das Individuum als Mensch in den Vordergrund.

\section{Kirchengeschichte}
Zum Verständnis und zum Hinterfragen der heutigen Tauftheologie ist es hilfreich, die Entwicklung des Rituals über die Zeit zu beachten.

\subsection{Taufverständnis in den ersten drei Jahrhunderten}
In vorkonstantinischen Zeit gab es noch keine einheitliche Tauftheologie. Noch bis ins 5. Jahrhundert lassen sich große Differenzen selbst in der Taufliturgie feststellen. Charakteristisch für den Umgang mit Taufe im 2. Jahrhundert ist z.B. Justins Apologie, laut der eine Taufe nur möglich ist, wenn der Täufling die Botschaft des Evangeliums zustimmt, danach die christliche Lebensweise annimmt und der Täufling bekommt eine Sündenvergebung von Gott zugesprochen. Dies stimmt auch mit den Schlussfolgerung aus den biblischen Zeugnissen überein.
\\
Der Taufritus bis ins 2. Jahrhundert hinein waren Wassertaufen und wurden als ein einmaliger Initiationsritus gesehen. Getauft wurde auf den trinitarischen Namen. Die Betrachtung als Versiegelung kam auf und weiterhin war die Taufe mit einer Geistverleihung verbunden. Interpretiert wurde die Taufe als eine Wiedergeburt und als Erleuchtung, was als eine Neuschaffung von Gott betrachtet werden kann.

\subsection{Tauftheologie im 4.\ und 5. Jahrhundert}
Im 4.\ und 5. Jahrhundert entwickelten sich zwei Arten von Tauftheologien: Die östliche und die westliche. In der östlichen Theologie wird die Taufe als aktiver Vorgang, in den der Täufling gegenüber der Sünde abstirbt und einen Prozess der Vervollkommnung und Heiligung durchschreitet, verstanden und die Taufe wurde somit zum Abbild des Leidens Christi gesehen. Folglich musste eine Geistsalbung folgen.
\\~\\
In der westliche Theologie wurde es als eine Neuschöpfung des Menschen gesehen, sodass mit der Taufe der Prozess abgeschlossen und somit keine Salbung notwendig ist. Sakramente wurde in Zeichen (Signum) und Sache (res) getrennt. Daher gilt laut der Sache die Taufe unabhängig von der Würde des Täufers. Das Zeichen konnte durch eine korrekte Zeichenhandlung repräsentiert werden, die nur in der Kirche vollzogen werden kann.

\subsection{Pietismus und Rationalismus in der Zeit der Aufklärung}
In der Zeit der Aufklärung hat sich die Taufpraxis nochmal entscheidend geändert. Die Taufe wurde immer mehr als individuelles Ritual im Kontext eines subjektiven Glauben gesehen. Es wurde nicht mehr ein äußerlich korrekter Vollzug der Riten, sondern ein innerer Prozess im Gläubigen  gefordert. So wurde die Taufliturgie bereinigt, wie z.B. traditionelle Elemente, die in Verbindung mit Aberglauben gebracht werden konnten. Auch wurde die Taufe nicht mehr nur in der Kirche vollzogen, sondern z.B. auch Zuhause.
\\~\\
Weiterhin galt die Geistverleihung als Konsens, jedoch dürfe die nicht zu einer Befreiung des neuen, individuellen Lebenswandels verleiten. In diesen Kontext rückte auch die Konfirmation als Element der Tauferneuerung, Bekennung, Verpflichtung und mündig werdens in den Vordergrund. Aus diesen Gedanken heraus wurde auch die Kindertaufe abgelehnt.

\subsection{Sonderformen der Taufe}
In der Kirchengeschichte gab es verschiedene Sonderformen der Taufe, von denen einige hier angeführt werden sollen. Diese weichen von der Taufpraxis, die auf die Johannestaufe zurückzuführen sind, ab.

\subsubsection{Taufe der Toten}
Zu Zeiten der Verfassung des Korintherbriefes gab es die Praxis einiger Christen, sich stellvertretend für die Toten taufen zu lassen, falls diese sich zu Lebzeiten nicht selber taufen lassen konnten. Paulus spricht diese Praxis an und toleriert diese zumindest, korrigiert diese Praxis aber nicht explizit. Die dahinter liegende Frage, ob Taufe als Bedingung für die Teilhabe an Jesu Heilstat zwingen erforderlich ist oder nicht, wird später beantwortet. Aufgrund der theologischen Fragwürdigkeit wurde diese Form aber nicht weitergeführt.

\subsubsection{Bluttaufe}
Die Bluttaufe war im 2. Jahrhundert anerkannt und vertreten. Zu Zeiten der Christenverfolgung wurde die Bekennung zu Christus und das darauf folgende Martyrium als Taufe anerkannt, da dort offensichtlich eine klassische Taufe nicht möglich war.

\subsubsection{Selbsttaufe}
Im 3. Jahrhundert waren auch Selbsttaufen durch z.B. Aufsagen der Taufformel und Stürzen ins Wasser als gültig anerkannt.

\subsubsection{Unmündigentaufe (Säuglingstaufe)}
Einigen könnte diese Einordnung komisch vorkommen, jedoch ordne ich sie als (bestreitbare) Sonderform ein, da sie nicht aus den biblischen Zeugnissen entspringt und es einige theologische Streitfragen dahinter gibt.

\subsection{Zwischenfazit zu der kirchengeschichtlichen Entwicklung der Taufe}
Es ist eine klare evolutionäre Entwicklung des Taufrituals und der Taufdogmatiken zu sehen. Eindeutig ist auch
der Einfluss von gesellschaftlichen, (kirchen-) politischen und andere theologische Grundthemen auf das Taufverständnis zu sehen.
Die Diskussion der Taufthematik hat sich über die Laufe der Zeit daher von der reinen Taufüberlieferung aus der Bibel
herausgelöst. Zu keiner Zeit hat sich eine Taufdogmatik als die einheitlich gültige Dogmatik herauskristallisiert, außer sie wurde
absolutistisch durch Autoritäten durchgesetzt. Immer wieder gab es Reformationen im Bezug auf die Taufthematik.
\\~\\
\newpage
Die heutige Taufdogmatik  darf daher nicht absolutistisch und als die eine Wahrheit gesehen werden, die sich rein auf die Überlieferung der Bibel begründet. Entweder muss daher die Thematik unter Berücksichtigung von aktuellen gesellschaftlichen und kirchenthematischen Themen komplett neu gesehen werden oder die Diskussion ist von all dies abzukapseln und man muss sich auf die ersten Überlieferungen zurückbesinnen. Bei ersteren ignoriert man jedoch den gesellschaftlichen und religiösen Einfluss auf die Thematik bei der Entstehung der Taufe, daher ist ersterer Ansatz zu wählen oder bei letzteren Ansatz die Taufe als ``theologisches und künstliches Produkt'' zu betrachten. Für diese Arbeit wird der zweite Ansatz gewählt.
\\~\\
Jedoch hat sich aus der Geschichte auch ein Konsens gebildet, der vermutlich für immer weitergelten und sich in allen Taufdogmatiken wiederspiegeln wird: Das Element der Geistverleihung, der Gedanke des Initiationsritus, die dafür notwendige innere Haltung und Verpflichtung auf ein Lebenswandel hin zu Gott, das Wasser und die grobe Form des ``Taufbades'' als äußeres Ritual, die Sündenvergebung, dass der Täufer nur eine Art mediatorische Rolle einnimmt und am Ende die Geistverleihung immer von heiligen Geist ausgeht und keinen anderen. Diese Themen werden sich daher auch immer wiederfinden und daher werden diese Themen weitergehend behandelt, sofern diese nicht selbsterklärend sind.


\section{Betrachtung der Kernaspekte}
\subsection{Taufe als Beitritt zur weltlichen Glaubensgemeinschaft}
Die Taufe symbolisierte schon immer den Eintritt in die Glaubensgemeinschaft. Egal ob bei Johannes, den Urgemeinden oder in der Kirchengeschichte: Die Taufe war immer ein öffentliches Bekenntnis zum Glauben und als weltliches Zeichen resultierte daraus der Beitritt in die Glaubensgemeinschaft. Jedoch war in der Kirchengeschichte und heutzutage an manche Orten immer noch die Glaubensgemeinschaft eng mit der Kirche als Institution verbunden. Eine getrennte Betrachtung von Kirche als Institution und Kirche als Glaubensgemeinschaft war damals nicht gegenwärtig.
\\~\\
Heute gibt es weder ``die eine Kirche'' noch muss eine Glaubensgemeinschaft zwingend in einer Kirche als Institution verortet sein. Jedoch war es zur Zeit der biblischen Überlieferung die Situation nicht anders. Die Getauften kamen aus diversen Gemeinschaften. Entweder aus der institutionellen Gemeinschaft der Juden, aus fremden Völkern oder komplett heidnischen Glaubensgemeinschaften. Die Taufe erfolgte bei allen zu den Zeitpunkt der Entscheidung, sich den Glauben an Jesus (später in seiner dreieinigen Form) und daraus resultierend seiner Gemeinschaft anzuschließen und sein Gnadengeschenk anzunehmen. Jedoch herrschte kein institutionalisiertes Verständnis der Gemeinschaft vor. Man organisierte sich mit gleichgesinnten und
mit Glaubensgeschwistern. Man organisierte sich in Hausgemeinden. Die Taufe war ein äußeres Zeichen für die Gemeinschaft vor Ort und trotzdem herrschte das Bewusstsein
über die Ortsgemeinschaft hinaus als Corpus Christi.
\\~\\
Heute kann man die Analogie ähnlich sehen. Die Art der Glaubensgemeinschaft spielt keine Rolle. Wichtig ist die Zugehörigkeit zum Christentum, aber nicht zu einer Kirche als Institution. Aus den Hausgemeinschaften werden Konfessionen, Freikirchen, freie (und lose) Glaubensgemeinschaften, Hauskirchen usw.. Das Bewusstsein
über die eigene Ortsgemeinschaft hinaus ist die Gemeinschaft zu der weltweiten, nicht institutionellen christlichen Gemeinschaft. Nach meinen Verständnis verkörpert die Taufe die Bekennung
\& der Beitritt zur weltweiten, christlichen Gemeinschaft und das öffentliche Zeichen gilt der Gemeinschaft vor Ort, mit der man den Glauben leben will. Daher
sollte auch der Beitritt einer Institution keine Verpflichtung sein, solange man sich der großen Glaubensgemeinschaft zugehörig fühlt. Vielmehr kann es ein Zeichen der Ökumene sein, da man sich mit
der Taufe den Beitritt der weltweiten christlichen Gemeinschaft bekannt hat und das beinhaltet alle Christen.
\\
Ebenso sollte bei einen Wechsel der Institution z.b.\ von evangelisch zu katholisch die Taufe keine Rolle spielen, da sie ja schon im Sinne der weltweiten christlichen Gemeinschaft vollzogen wurde
und man der Gemeinschaft der Christen schon beigetreten ist. Ebenso darf allein aufgrund der Tauffrage jemand die Zugehörigkeit zum Christentum nicht an- oder aberkannt werden.
\\~\\
Aufgrund dessen, dass in den biblischen Texten das Taufritual erst in seiner Findung war und besonders strittige Formen der Taufe (namentlich: Säuglingstaufe) noch nicht existierten, gibt es kein biblisches Vorbild wenn es um das Thema der ``vermeintlich falschen Taufe'' geht. Jedoch ist ein Schema zu erkennen: Bei Zweifel wurde dies vor der Taufe geäußert in der Form, dass jemand noch nicht zur Taufe bereit war (siehe Johannes \& die Pharisäer oder Johannes \& Jesus), jedoch nicht nachdem die Taufe vollzogen wurde. Es wurde höchstens behauptet, dass die Taufe keine Wirkung aus der eigenen Perspektive heraus hatte. Sie haben aber nicht den anderen die Taufe aberkannt. Aus meiner persönlichen Sicht heraus ist dies ein schönes und durchaus ökumenisches Bild: Die Zeit der persönlichen Debatte ist vor den Taufritual zu verorten. Falls wir bei bereits erfolgter Taufe die andere Taufverfassung nicht vertreten, dann erkennen wir persönlich nicht die Wirkung dieses Taufrituals an und setzten es optional für uns persönlich heraus z.B. zu einer Segnung herab. Wir gehen aber nicht zu den getauften hin und versuchen, ihn seine Taufe abzuerkennen.

\subsection{Der Glaube geht dem Taufritual voran}
Des weiteren stellt sich die Frage, was zuerst kommt: Das christliche Leben mit den Glauben, das in die Taufe resultiert oder die Taufe, die das christliche Leben beginnt? 
\\~\\
Hierbei spielt ausgerechnet die Argumentation mancher (katholischer) Befürworter der Kindertaufe in die Hand. Augustinus begründet die Säuglingstaufe mit der Erbsünde, welche mit der Taufe vergeben werden könnte.
Wer aber hat nicht mehr die alleinige Macht, uns von der Erbsünde loszulösen, als Jesus Christus in seinen Geschenk des Kreuzestodes?
Durch seinen Kreuzestod hat er uns ein Geschenk gegeben, welches wir aber annehmen müssen. Um es annehmen zu können muss aber als Konsequenz ein aktives Annehmen dieses Geschenkes gefordert werden. Hierfür muss der Glauben an Jesus und sein Geschenk vorangehen. Analog muss auch der Glauben der Taufe vorrausgehen, wenn man mit der Erbsünde argumentiert.
\\
Schleier und Bonhöfer sehen in der Kindertaufe ein Vertrauen in Jesus Christus und der Glaubensgemeinschaft, den Täufling zum
Glauben zu bewegen. Am Ende muss jedoch der Täufling in Form der Konfirmation oder ähnliches die Taufe bestätigen. Die Taufe wird erst dann vollzogen. Eben diese Bestätigung zeigt aber wieder rum, dass eine Unmündigentaufe keine wahre Taufe sein kann und daraus der Glaube der Taufe vorangehen muss. Es gibt noch viele andere Versuche, die Säuglingstaufe biblisch zu legitimieren, jedoch ähnelt dies immer einen Versuch, einen \underline{einzigen} Beleg zu finden, um gegen die Fülle der Belege einer Glaubenstaufe anzukämpfen. Eine Empfehlung ist hier das Buch ``Der Streit um die Taufe'' (Klaus Hoffmann), das es schafft alle potentielle ``Grashalme'' zu einer Glaubenstaufe hin zu deuten und die Unmündigentaufe unplausibel zu machen oder je nach Perspektive komplett zu wiederlegen. Am Ende beraubt man aber auch den Säugling seiner Tauferfahrung mit der Unmündigentaufe, daher ist von meiner Seite eine Taufe der unmündigen abzulehnen. Eine alternative zur Säuglingstaufe stellt die Kindersegnung dar, die als Zeichenhandlung primär den Eltern und der Glaubensgemeinschaft zugute kommt, den Säugling jedoch nichts nehmen.
\\~\\
Am Ende ist aber klar, dass der erweckte Glaube und die Taufe auf symbolische Ebene eng verbunden sind. Einer Taufe geht das Handeln Gottes und daraus resultierend die Erweckung des Glaubens voraus.
Aber erst durch die Bekennung durch das Zeichen des Taufrituals wird der Prozess des Gläubigwerdens vollendet, weil dies den Wille zur Nachfolge symbolisiert. 

\subsection{Das Taufritual bringt nicht das Heil}
Am Ende ist das Taufritual ein Zeichen der weltlichen Bekennung zum Glauben und zur Glaubensgemeinschaft. Man muss jedoch aufpassen, die Zeremonie nicht zu überhöhen.
Das Wirken Jesu ist durchzogen von den Aufruf, sein Leben Gott und somit Jesus zu verschreiben. Jedoch bindet Jesus seine Sündenvergebung und den Glauben
weder an das Taufritual noch an andere Rituale. Er bindet es allein an sich selber und an seine Gnade. Nur weil Jesus aller Wahrscheinlichkeit nach das Taufen
akzeptiert und sich selbst der Taufe durch Johannes unterzogen hat, heißt das nicht dass er dies an sein Heil gebunden hat.
Das Taufen ist ein Ritual und Zeichen eines Gläubigen an die Glaubensgemeinschaft und \underline{persönlich} an Gott. Bekennung in Form eines Taufrituals hat viele Formen
und man darf nicht in Versuchung verkommen, eine bestimmte Form der Taufe als ein elementares Zeichen an den Glauben selber zu sehen. Vielmehr ist die Taufe und darin enthalten das
Konzept der Geistverleihung, Sündenvergebung und Zusprechung des göttlichen Heils ein symbolisches, weltliches Ritualzeichen, welches durch Gottes Handeln seine volle
Wirkung entfaltet oder eben ein weltliches Zeichenritual bleibt.
\\
Wie sich vermehrt herausstellt, benötigt das Glauben, die Geistesverleihung und die Sündenvergebung das Handeln Gottes. Das Taufen als weltliches Ritual kann
nur seine Wirkung durch Gott entfalten. Und ein Gott, der in Form seines Sohnes am Kreuz den Gekreuzigten neben ihn das Heil zugesprochen hat ohne Taufe o.ä.
zeigt, dass man sich nicht anmaßen darf, das Taufritual als absolute Bedingung für das Heil zu sehen. Ebenso wäre eine Bindung der Taufe an ein kirchliche Autorität,
Institution oder Absolutismus eine Anmaßung an Gottes handeln selber.
\\~\\
Dies soll jedoch nicht die Tauferfahrung des Täufling aberkennen. Rituale und Sakramente sind wichtig (mehr dazu im Abschnitt\ \ref{sec:Sakramente}: Sakramente und Rituale).
Das Taufritual ist biblisch und ein perfektes Ritual für den Gläubigen und der Gemeinschaft, die Bindung der Taufe an die Teilhabe des Heils jedoch nicht.
Am Ende ist das Taufritual ein perfektes erstes Zeichen des Gläubigen in seinen Weg der aktiven Jüngerschaft, welches zurück bis hin zu Jesus Zeit selbst geht. Eventuell wirkt Gott sogar durch ein aktives Handeln in der Taufe, vielleicht sieht er im Taufritual aber auch nur wie ich eine Zeichenhandlung.
Die enorme Entwicklung der Wichtigkeit der Taufe ist aber aller Wahrscheinlichkeit menschlicher Natur und sollte daher nicht überschätzt oder sogar als absoluter Wille Gottes verstanden werden.

\section{``Bildliches Fazit'': Taufe als umfassendes Konstrukt des Gläubigwerdens}
Die Taufdiskussion kann -erweitert zum Taufritual selbst- als ein Prozess des Gläubigwerdens verstanden werden, welcher mit der ersten Begegnung mit dem christlichen Glauben oder einer ersten Gotteserfahrung beginnt und mit dem Taufritual endet. Die Diskussion ist im strengen Sinn auch mit dem Prozess des Gläubigwerdens eng verbunden. Im folgenden sollen daher dieser Prozess und darin mit inbegriffen wesentliche Aspekte der Taufdiskussionen kurz \& anschaulich abschließend beschrieben werden und eine Art Zeugnis des eigenen Taufverständnisses darstellen.
\\~\\
Der ganze Prozess beginnt mit einer ersten Begegnung mit den christlichen Glauben durch z.B. Mission,Verkündigung oder einer ersten Gotteserfahrung. Dabei kann sprichwörtlich der Stein vielseitig zum Rollen gebracht werden und muss nicht nur als ein stereotypisches Bild der Mission gesehen werden. Das historische Bild ist hierbei sicherlich auch eine christliche Erziehung durch die Eltern, jedoch kann durch den demographischen Wandel und durch die Austrittszahlen usw.\ dies nicht ewig als Grundlage gesehen werden. Daher ist es wichtig, in Zukunft Mission und Verkündigung neu zu denken, um Ungläubige den Glauben näher zu bringen und den Prozess des Gläubigwerdens anzustoßen.
\\~\\
Darauf folgt ein höchst individueller und nicht generell (kurz) beschreibbarer Prozess, in den unter anderen auch der Heilige Geist wirkt und durch den der ``Gläubigwerdende'' zu folgenden Einsichten gelangt:
\begin{itemize}
	\item Er/Sie erkennt Gott in seiner dreieinigen Form als allmächtigen, allwissenden und barmherzigen Vater an
	\item Er/Sie baut eine Sehnsucht auf, Teil von Gottes Reiches zu werden
	\item Er/Sie erkennt, dass er/sie niemals aus eigener Kraft heraus würdig genug sein kann für Gottes Reich
	\item Er/Sie erkennt, dass Gott durch Jesu Wirken am Kreuz uns eine Gnade geschenkt hat, die uns das Reich Gottes würdig macht und dass wir jederzeit annehmen können
\end{itemize}
Durch diese Einsichten nimmt der ``Gläubigwerdende'' dieses Gnadengeschenk an. Dieses Annehmen befreit uns von der Erbsünde \& unseren Sünden der Vergangenheit und eröffnet die Möglichkeit, in Zukunft immer wieder zu Jesus zu kommen und unsere neuen Sünden ihn anzuvertrauen und somit abzugeben.
\\
Dieses Gnadengeschenk bewegt den ``Gläubigwerdenden'' als \textbf{Konsequenz} (und nicht aus Zwang, Gesetz, Vorraussetzung o.ä.) zu einen Leben mit Gott in seiner Dreieinigen Form. Zu diesen Zeitpunkt wird/ist der ``Gläubigwerdende'' zum gläubigen Christen geworden.
\\~\\
Der Prozess wird mit dem Taufritual abgeschlossen, mit dem sowohl der Gläubige als auch (wie bei der Taufe Jesu) Gott in Form des heiligen Geistes diesen Prozess abschließen und bestätigen. Mit diesen Taufritual beginnen der Gläubige und Gott ein neues gemeinsames Leben, was in vielen Kreisen auch Jüngerschaft genannt wird.
\\~\\
Dieser geschilderte Prozess ist nur ein Modell und bietet natürlich an vielen Stellen Freiräume zum ausbrechen oder für zusätzliche Themen oder Diskussionen. Das grobe Wirkungsgefüge jedoch bleibt gleich nach meiner Auffassung immer gleich. Aus diesen Wirkungsgefüge (und vorigen Erkenntnissen) gehen folgende Kernaussagen zur Taufe heraus:
\begin{itemize}
	\item Die Erbsünde und alle Formen von Sünden werden vor dem eigentlichen Taufritual allein und ganz allein durch Jesus Christus Heilsgeschehen am Kreuz vergeben. Der Täufling nimmt dieses Gnadengeschenk an und partizipiert somit am Heilereignis.
	\item Dem Taufritual geht immer der eigene Glaube vor raus
	\item Das Taufritual ist einmalig, da man nur einmalig sich zu einen Leben mit Gott entscheiden kann. Alle weiteren, erneuten Entscheidungen zu einen Leben mit Gott sind z.B. ein zurückfinden zum eigenen Glauben und bedarf einen neuen/eigenen Ritual losgelöst vom Taufritual. Hierfür könnten sich eine Form der Segnung anbieten.
	\item Das Taufritual hat einen klaren bekennenden Charakter
	\item \noindent\fbox{\parbox{\textwidth}{
			      Das Taufritual ist sowohl ein Bekenntnis vor Gott als auch der Glaubensgemeinschaft.\ \textbf{Daher liegt es an den Getauften, ob sein Taufritual die ``richtige Taufe'' vor Gott verköpert und nicht an der Glaubensgemeinschaft/Instutition etc.!} Die Glaubensgemeinschaft hat keine Form der Taufe vorzuschreiben und kann seine Form der Bekenntnis o.ä.\ in z.B. eine Form der Konfirmation von der Taufe losgekoppelt ``einfordern''. Ein gemeinsames Taufverständnis und damit verbunden ein gemeinsames Bild des Taufrituals der Glaubensgemeinschaft und des zu Taufenden wären hierbei natürlich der beste anzustrebende Zustand.\ \textit{Daraus resultierend ist eine Kindertaufe, an die der Getaufte nicht glaubt, nichts mehr als eine Kindersegnung und die vermeintliche Wiedertaufe ist eine richtige, erstmalige Taufe.}
		      }}
	\item Das Taufritual ist nicht gebunden an spezielle Voraussetzungen an den Täufer oder an eine Institution o.ä.
	\item Der rituelle Ablauf des Taufrituals ist nicht ``heilig'' oder Gott gegeben, sondern ein Menschenwerk, in dessen Gott sein Wirken hineinsetzt, aber keine strikten Vorgaben gibt. Der biblische rituelle Ablauf ist ein vollständiges Untertauchen in fließenden Gewässer (Fluss, Bach o.ä.).
	\item Das Taufritual beinhaltet viele theologische Motive (Wegwaschen der Sünden, Taufen auf den trinitarischen Namen, das dreimalige Fragen und Bekennen etc.) aus dem Prozess des Gläubigwerdens, dass in das Ritual bildlich wiederzufinden ist. Das Taufritual gibt diese Elemente symbolisch wieder, jedoch wohnt den eigentlichen Taufritual kein besonderes Wirken bei, vor allem keins was nur diesen speziellen Event zuzuordnen wäre (außer das es einmalig im Leben ist)
	\item Alle alleinige Konsequenz aller Tauferkenntnisse: Das Taufritual bewirkt nicht das Heil, gibt nicht die ewige Seligkeit und ist darum nicht heilsnotwendig!
\end{itemize}

\section{Fazit}
Die Taufe ist biblisch überliefert. Es ist auszugehen, dass Jesus diese akzeptiert hat, eine direkte Verbindung aller Kerninhalte zu Jesus ist jedoch nicht einwandfrei nachzuweisen. Dabei kann das Taufthema in die Taufe als Prozess des Gläubigwerdens und in das Taufritual selbst aufgespalten werden.
\\~\\
Taufe ist mit der Geistverleihung verbunden, ohne die Geistesgabe an die Taufe zu binden.
Das Taufritual ist daher kein absolutistisches Kriterium für die Teilhabe am Heil, da die Taufe durch Jesus, die Geistestaufe oder das Wirken des heiligen Geistes  -welches von den Modellen man auch immer nehmen will- über der Wassertaufe steht. D.h.\ die Taufe ist ein Prozess, welche den Taufenden dazu verleitet, Gottes Gnadengeschenk anzunehmen und ihn somit noch vor dem Taufritual an Gottes Heil teilhaben lässt. Am Ende ist das Taufverständnis ein persönliches Verständnis und eine Sache zwischen Gott und den Täufling. Daher kann nur der zu Taufende entscheiden, ob eine Taufe auch ``seine persönliche Taufe'' war. D.h.\ das Bild einer Wiedertaufe in Sinne einer zusätzlichen Taufe zu einer bereits vorhandenen Unmündigentaufe gibt es nur vonseiten einer Institution und nicht vonseiten des zu Taufenden.
\\~\\
Das Taufritual ist ein einmaliges, weltliches Ritual und Zeichen,
das mit den Beitritt zur übergeordneten christlichen Gemeinschaft zuzuordnen ist. Das Taufritual markiert somit den ersten Schritt zur aktiven Jüngerschaft. Dem Taufritual geht der Glauben voraus
und ist eine Bekennung hin zu den christlichen Glauben und Bezeugung des Willens zur Nachfolge vor der weltlichen Gemeinschaft \& Gott. Daraus resultiert zwingend die Mündigkeit des Täuflings, eine Unmündigentaufe ist daher eher abzulehnen
und sollte eher als eine Art Segnung betrachtet werden. Es gibt nicht das eine, korrekte Ritual dahinter. Das biblisch überlieferte Ritual ist das vollständige Eintauchen in fließendes Gewässer d.h. Flüsse und Bäche. Das Taufritual kann von Menschen
mediatorisch vollzogen werden nach besten Gewissen und Wissen, aber am Ende kann der Mensch weder die Sündenvergebung noch andere vorgestellte Elemente aktiv bewirken, sondern nur Gott selber. Der Täufer nimmt daher eine untergeordnete Rolle ein. Am Ende ist das Taufritual als weltliches Ritual zur Beendigung des Prozesses des Gläubigwerdens jedoch nichts entgegenzusetzen.
\\~\\
\textbf{Persönliche Anmerkung und böse formulierter Apell}: Sich an dem Taufthema aufzureiben, dem Taufenden ein persönliches Taufverständnis vonseiten einer Institution/Glaubensgemeinschaft aufzuwingen und wegen verschiedenen Taufverständnissen sich gegenseitig zu exkommunizieren/umzubringen/auszuschließen ist Bullshit und grenzt an eine Beleidigung des Wirkens Gottes!

