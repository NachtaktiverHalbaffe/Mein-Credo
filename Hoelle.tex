% !TEX root = Theologische_Grundfragen.tex
\chapter{Hölle}
In der Bibel gibt es drei Wörter für die Hölle: Scheol, Hades und Gehenna. Scheol stammt aus den Alten Testament und meint so etwas wie das Totenreich.
Dieses Wort ist hierbei bei den eher poetisch einzuordnenden Texten vorzufinden. Hierbei wird das Wort oft als Synonym wie für Tod, Grab oder Grube verwendet.
Somit war das Scheol erstmal der Tod selbst und er herrschte kein Bild vor, in den die unsterbliche Seele in irgendeiner Weise weiterexisitert.\\

Durch den Einzug des persischen Dualismus wurde neben den Engel, Teufel und Dämonen auch die apokalyptischen Lehren im Judentum geboren. In dieser wird wie in Kapitel\ \ref{sec:Teufel} erläutert die Schuld des Bösen an die bösen Mächte abgeschoben und die Idee des jüngsten Gerichts eingeführt, in der Gott die Gerechten belohnt und die Ungerechten bestraft. Für diesen Zweck werden die Körper der Toten wiedererweckt, also aus dem Scheol zurückgeholt.
Hierbei musste es als eine Erweckung des Körpers verstanden werden, da in jüdischer Tradition nur ein Körper leben kann. Für die gerechten wird es eine gerechte Welt ohne Leid und Sorgen geben und für die Ungerechten eine Bestrafung. Ein dualistisches Denken, in dem es unterschiedliche Totesschicksale für Ungerechte und Gerechte gibt oder eine Erettung aus den Gottesreich möglich ist, gibt es selten.  Jedoch gehört hier auch das Totenreich in Gottes Machtbereich.\\

Im neuen Testament wird von der Hölle als Hades und von Gehenna gesprochen. Ersteres ist die griechische Übersetzung des Wortes Scheol und meint somit dasselbe Totenreich. Letzteres ist schlicht ein Tal neben Jerusalem, in dem offenbar in einen Kult Kinder und andere Sachen als Opfer verbrannt wurden und als Müllhalde mit vielen Toten Tieren, Kadavern etc, diente. An diesen Ort wurden die Leichen von Toten entsorgt, wenn diese nicht beerdigt werden konnten
oder durften. So beerdigt zu werden war für die damaligen Menschen eine furchterregende Vorstellung. Dieser Ort war somit quasi ein \glqq Hölle auf Erden\grqq{}, welcher sich jedoch auf Erden befindet und kein übernatürlicher Ort ist. Wenn Jesus von Gehenna und Hades spricht, so meinte er nach jüdischen Verständnis das jüngste Gericht, in den die Ungerechten zwar nicht ewig gefoltert werden, aber deren Existenz nach dem Gericht ausgelöscht wird.