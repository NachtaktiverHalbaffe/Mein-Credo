% !TEX root = Lehren__von_Jesus_Christus.tex
\chapter{Versuchung} \label{Versuchung}
Nach seiner Taufe zog sich Jesus nach den Evangelien von Matthäus, Markus und Lukas in die Wüste zurück, um den Versuchungen des Teufels zu widerstehen. Deshalb soll im folgenden Kapitel das Thema „Versuchung“ näher betrachtet werden. Da es durch den Aspekt des Bösen eng mit der Theodizee-Frage verbunden ist und beide Themen eine gemeinsame, grundlegende Fragestellung berühren, wird an dieser Stelle zugleich auf das Kapitel Theodizee verwiesen (siehe Kapitel \ref{Theodizee}).\\

In der theologischen Literatur wird das Thema Versuchung nur selten ausführlich behandelt. Zwar findet es im Vaterunser Erwähnung, doch in vielen theologischen Enzyklopädien wird es gar nicht aufgeführt. Wo Versuchung thematisiert wird, geschieht dies oft recht oberflächlich: Meist beschränkt man sich auf eine kurze Definition und stellt anschließend dogmatisch fest, ob auch Gott versucht oder nicht. Auf intellektueller Ebene scheint Versuchung für viele also keine zentrale Rolle zu spielen; sie taucht zwar als Gegenstand von Diskussionen auf, wird jedoch selten selbst zum eigentlichen Diskussionsthema.\

Ganz anders verhält es sich in den persönlichen Glaubenserfahrungen. Hier nimmt die Versuchung eine bedeutende Rolle ein: Viele Gläubige deuten ihre Herausforderungen, Probleme und Leiden als Formen der Versuchung, oft verbunden mit Vorstellungen von Teufel, Dämonen oder bösen Mächten (vgl. Kapitel \ref{sec:Teufel}).\

Daraus ergibt sich eine deutliche Diskrepanz zwischen der theologisch-dogmatischen Behandlung der Versuchung und ihrer existentiellen Bedeutung in der Glaubenspraxis. Die erste Schlussfolgerung lautet daher: Aus theologischer Sicht spielt die Versuchung bislang keine wesentliche Rolle als eigenständiges Thema – wohl aber erweist sich in der persönlichen Glaubenserfahrung ein gesundes und reflektiertes Verständnis von Versuchung als dringend notwendig.\\

Das Ziel dieser Arbeit besteht daher nicht darin, das Thema in seiner ganzen Tiefe theologisch auszuschöpfen. Vielmehr soll erläutert werden, was unter Versuchung zu verstehen ist, wie sie sich konkret äußern kann und welche praktischen Hilfen sich daraus für die Glaubenspraxis ableiten lassen. \\

Der Begriff Versuchung leitet sich vom griechischen Wort peirasmos ab und bedeutet „Prüfung“, „Lockung“ oder „Anfechtung“. Im theologischen Sinn kann Versuchung daher als Bewährungsprobe im Verhältnis zwischen Mensch und Gott verstanden werden. Dabei lässt sich zwischen Versuchungen durch Gott, durch Mitmenschen und durch die eigene Begierde unterscheiden. Versuchungen durch Dämonen oder den Teufel hingegen können vernachlässigt werden, da deren Existenz in diesen Credo angezweifelt wird und entsprechende Phänomene letztlich auf die Unvollkommenheit des Menschen und seine Neigung zum Bösen zurückgeführt werden können. \\

Die Versuchung durch den Menschen zeigt sich auf zweierlei Weise: Sie kann sowohl durch andere Menschen hervorgerufen werden als auch aus den eigenen Begierden entstehen. Nahezu alle Versuchungen sind in diesem Sinn menschlich verursacht und unabhängig von einem direkten Handeln Gottes. Auch die Bibel legt vergleichsweise wenig Gewicht auf alltägliche Versuchungen, die unmittelbar von Gott oder – metaphorisch gesprochen – vom Satan ausgehen. Gerade menschliche Versuchungen können jedoch eine Belastung darstellen, die weit über das Erträgliche hinausgeht, da sie von uns selbst oder unseren Mitmenschen unbegrenzt auferlegt werden können. Häufig betreffen sie grundlegende Bedürfnisse des Menschen (ohne Anspruch auf Vollständigkeit): Essen, Schlaf, materielle und berufliche Sicherheit, Anerkennung, Selbstverwirklichung und Sexualität. Diese Bedürfnisse erweisen sich immer wieder als Einfallstore für destruktives Verhalten und werden daher oft als Ansatzpunkt des Bösen verstanden.\\

Die Versuchung durch Gott hingegen verfolgt einen anderen Zweck: Sie soll den Glauben prüfen und stärken, den Menschen näher zu Gott führen und seine innere Standhaftigkeit offenbaren. Zugleich betont die biblische Überlieferung, dass Gott den Menschen in der Versuchung beisteht und ihm nicht mehr aufbürdet, als er tragen kann. Darin liegt auch das wesentliche Unterscheidungsmerkmal zur Versuchung durch den Menschen: „Böse“ Versuchungen kommen nicht von Gott. Gottes Prüfungen zielen vielmehr auf Vertrauen und Loyalität. Auffällig ist jedoch, dass Berichte über eine derartige Anfechtung im Alltag selten sind. Zu den bekanntesten Beispielen zählen die Versuchung Hiobs – ein poetisches Werk, das ein einzigartiges Geschehen schildert – sowie die Versuchung Jesu, die im folgenden Abschnitt näher betrachtet wird. Es liegt daher nahe, dass Versuchungen durch Gott im Glaubensleben vieler Menschen eher selten vorkommen. Der eigentliche Trost und Gewinn dieser Erzählungen besteht vielmehr darin, dass Gott seine Gläubigen inmitten ihrer – meist menschlich verursachten – Versuchungen begleitet, sie hindurchträgt und ihnen am Ende in Treue begegnet.

\section{Versuchung von Jesus}
Auch Jesus wurde in Versuchung geführt, und zwar während seiner Wanderung in der Wüste (wahlweise Einöde). Die Erzählweise dieser Geschichte entspricht einem Muster, das bei wiederkehrenden Lebensereignissen genutzt wird, die nicht jedes Mal neu geschildert werden mussten. Daraus folgt, dass diese Episode nicht unbedingt als historisches Ereignis, sondern vielmehr als gesammelte (Lehr-)Erzählung über Jesus verstanden werden kann, basierend auf den Überlieferungen seiner Begleiter. Bestätigt wird dies durch Lukas 22,28, wo Jesus von mehreren Versuchungen spricht, die er gemeinsam mit seinen Jüngern durchgestanden hat. Dies deutet darauf hin, dass Jesus wiederholt mit Versuchungen konfrontiert war, diese jedoch nicht vor seinen Jüngern verborgen hielt. Die Versuchungen in dieser Geschichte lassen sich daher als die zentralen Prüfungen verstehen, mit denen Jesus während seines Wirkens konfrontiert war, und zeigen gleichzeitig seine menschliche – und damit unvollkommene – Seite.\\

Es lassen sich drei unterschiedliche Versuchungen unterscheiden, die der Teufel – metaphorisch verstanden als die menschliche Neigung zum Bösen – Jesus entgegenbringt (Reihenfolge nach Matthäus):
\begin{enumerate}
    \item Versuchung durch die Grundbedürnisse: Verwandel Steine in Brot
    \item Versuchung zur Selbstoffenbarung: Beweise die eigene Gottessohnschaft durch Wunder
    \item Versuchung nach politischer Macht: Anbetung des Teufels im Austausch für Weltherrschaft
\end{enumerate}
Jesus sah sich mit diesen Versuchungen immer wieder konfrontiert und es war sicherlich nicht einfach, diese Abzuweisen.
Doch er sah sie immer als Abwege, indem er diese mit den Weisungen Gottes und seiner eigenen Glaubenslogik
widerlegte.\\

Die erste Versuchung betrifft die Grundbedürfnisse des Menschen. Dabei geht es nicht nur um Jesu eigene Bedürfnisse, sondern auch um die der Mitmenschen. Könnte er nicht den sozialen Wohlstand sichern, wenn er seine göttlichen Fähigkeiten nutzt? Jesus widersteht dieser Versuchung durch Vertrauen in Gott und das Bewusstsein, dass von Gott gegebene Gaben nicht zum Selbstzweck dienen, sondern zum richtigen Gebrauch.\\ 
Er vertraut auf Gottes Hilfe und widersetzt sich den Drang, seine eigene Hilfe schaffen zu wollen, zu welcher er mit seiner gegebenen Gottessohnschaft in der Lage wäre und in diesen Zusammenghang mit der zweiten Versuchung einhergeht: Seine Gott gegebene Macht und Gaben zu
missbrauchen. \\

In der zweiten Versuchung wird Jesus aufgefordert, Wunder zu vollbringen, um die Menschen von seiner Gottessohnschaft zu überzeugen. Der Satan argumentiert dabei mit Zitaten aus der Thora, was verdeutlicht, dass isolierte Bibelverse ohne Kontext irreführend sein können. Gottes Wort kann Versuchung und Lösung zugleich sein. Eine bittere Wahrheit, der sich jeder Christ -unabhängig seines Bibelverständnisses- klar werden muss. Jesus antwortet der Versuchung hierbei mit einen geschlossenen Bild seines Glaubens. Jesus begegnet dieser Versuchung mit einem kohärenten Glaubensverständnis, das die gesamte Schrift berücksichtigt. Wäre er diesem Anreiz erlegen, hätte er die Menschen nur oberflächlich überzeugt, nicht aber in ihren Herzen. Dies zeigt, dass diese Versuchung bestimmt verlockend war, aber nur ein Abklatsch dessen, was er durch sein Wirken erreicht hat: Eine Überzeugung, die vom Herzen kommt. Die Geschichte verdeutlicht, dass wahre Überzeugung intrinsisch ist, aus dem Inneren kommt und nicht durch äußere Zwangsmittel erzeugt werden kann. Zudem ist Jesus sich bewusst, dass Gott kein „Flaschengeist“ ist, der nach Belieben zu Diensten gerufen werden kann.\\

In der „letzten Versuchung” sieht sich Jesus mit der potenziellen politischen Macht konfrontiert, die er ergreifen könnte.
Soll er die Macht mithilfe politischer Mittel ergreifen? Würden die Menschen ihn besser verstehen, wenn er ihnen alles mithilfe politischer Macht näherbringt und notfalls einfach durchsetzt? Soll er die Machthaber verdrängen und die Menschen sowie neue Politiker für seine Sache begeistern? Jesus widersteht der Versuchung, indem er sich besinnt, dass diese nur die Herrschaft über die irdische Welt haben und er der Macht Gottes folgen will. Jesus sehnt sich nach dem Himmelreich Gottes und kann deshalb dem irdischen Reich widerstehen. Er ist sich bewusst, dass nicht Gott dir gehorcht, sondern du ihm, um dein Ziel zu erreichen. Er weiß, dass er in diesem Fall einfach von Gott abhängig ist und dem Plan Gottes vertrauen sollte, von dem Jesus nicht eigenmächtig abweichen sollte. Er sollte nicht einfach Macht ergreifen, wo Gott sie nicht vorgesehen hat.\\

Jesus war mit Versuchungen wie diese immer wieder konfrontiert. Jesus war wahrscheinlich sich nicht immer seines
Auftrages und seiner Gottessohnschaft so sicher, wie wir es uns vielleicht ausmalen. Dies zeigt diese Geschichte -v.a.\
im Kontext der anderen Geschichten, die darauf Bezug nehmen- auf. Der Unterschied ist, dass Jesus als Mensch wie wir
diesen Versuchungen auch ausgesetzt war, aber als Gottessohn diesen niemals erlegen ist. Hierbei hat er jedes mal auf
Gottes Wort -in Form der heiligen Schrift- vertraut und ansonsten keine Eigenleistung eingebracht, die über ein
Gottesvertrauen hinausging.

\section{Hilfe in der Versuchung}
Wir wissen, woher die Versuchungen kommen können und haben auch schon ein kronkretes Beispiel bei Jesus Versuchung in
der Wüste gesehen. Nachfolgend wird nun erschlossen, wie wir mit den Versuchungen umgehen und wie wir uns helfen
können. Hierbei wird von Versuchung durch den Menschen ausgegangen. \\

Zuerst geht es um die richtige Grundhaltung. In den Geschichten der Bibel hat immer das Vertrauen zu Gott derjenigen
Personen enorm verholfen. Dies war ihre Grundhaltung zu Gott. Solch ein Vertrauen muss natürlich erst aufgebaut werden,
daher ist es wichtig in seinen Glaubensleben an diesen Vertrauen zu arbeiten, sodass es uns dann in diesen Situationen
tragen kann. Bei den Geschichten aus der Bibel war es immer das Wort Gottes, auf dass dieses Vertrauen aufgebaut hat,
also den eigenen Glauben. Wir in unserer Lebensrealität müssen das finden, was für uns persönlich dieser aufbauende
Glaubensfaktor ist, da der Glaube mittlerweile mehr Zugänge zu Gott kennt. Dies kann also ebenso das Gebet oder
Lobpreis sein. \\

Viele der Versuchungen greifen an den Grundbedürfnissen des Menschen an. Daher ist es wichtig, sich stets gut um sich
selbst zu kümmern. Man sollte auf seine eigene Gesundheit achten, sein soziales Leben nicht vernachlässigen und
schauen, dass der sozialer Wohlstand für deine Bedürfnisse ausreicht, aber nicht dein Leben definiert. Hierbei war dies
nur ein Auszug aus den Bedürnissen. Grundlegend ist hier ein rationales und/oder reflektierendes Verhalten wichtig. So
können Versuchungen im voraus vermieden werden. Aber auch wenn man sich in einer Versuchung befindet, kann man schauen,
ob dieser auf eine Mangelerscheinung eines bestimmten Bedürfnisse zurückzuführen ist und die Versuchung durch das
Ausgleichen dieses Mangels sich von selbst löst. \\

Man muss nicht selbst durch eine Versuchung durch. Wenn man für die beiden vorangegangenen Punkte alleine nicht
zurechtkommt, dann darf und soll man sich Hilfe holen. Für das Gottesvertrauen kann man in die Glaubensgemeinschaft
gehen und dort mit den anderen reden, seine Erfahrungen teilen, miteinander beten und sich gegenseitig mentoren. Für
die Erfüllung von menschlichen Bedürfnissen kann man auch zu Hilfsangeboten gehen und dort helfen lassen. Dies ist
keine Schande. \\

Man muss sich aber auch bewusst sein, dass Dinge, die uns vermeintlich in der Versuchung helfen und herausführen
können, eigentlich Gift für uns oder das spezifische Problem sein können. Andere Menschen können auch Ratschläge geben,
die mehr Schläge als Rat sind oder uns in eine andere Versuchung locken. In Zeiten der Versuchung sind wir anfällig und
verletzbar und dessen muss man sich bewusst sein. Man kann in diesen Situationen -sei es bewusst oder unbewusst- durch
andere Menschen in der Versuchung bestärkt oder in eine neue Versuchung geführt werden. Auch der Teufel konnte die
Bibel zitieren. Auch die gut gemeinten Ratschläge oder Meinungen der Freunde von Hiob waren falsch. Daher muss man in
diesen Situationen auch ein ausgeprägtes Maß an Selbstreflektion und Skepsis mitbringen und dabei ein Gleichgewicht
bewahren, sodass diese nicht selbst zum Problem werden. Eine durchaus schwierige und nicht immer zu lösbare Aufgabe. \\

Zuletzt sei gesagt, dass die Menschen unterschiedlich mit Versuchungen umgehen. Es gibt kein Allzweckheilmittel und
viele Wege führen nach Rom. Es gibt freilich Dinge, die eine großer Gruppe von Menschen helfen, aber am Ende muss man
auch immer herausfinden, was selbst für einen Gut ist und was für einen selbst zutrifft.

\pagebreak
\section{Fazit}
Versuchungen sind Herausforderungen und Probleme, die unser Glauben und unsere Persönlichkeit auf die Probe stellen und
somit prüfen können. Ob Gott selbst in Versuchung führt, ist nicht auszuschließen, es kann aber angenommen werden, dass
dies in 99,99\% der Fälle definitiv nicht der Fall ist. Klar ist: Wenn Gott versucht, dann kommt nichts Böses von Gott.
Im Regelfall tritt die Versuchung durch den Menschen auf, entweder durch die Mitmenschen oder die eigenen Begierden.
Hierbei hat die Versuchung durch den Menschen selbst keinen Zweck im Bezug auf den Glauben, sie tritt nur auf. Hierbei
war auch Jesus von Versuchungen betroffen. Dies zeigt einerseits die Menschlichkeit des Gottessohnes wiederholt auf und
zeigt andererseits, dass man sich für seine Versuchungen nicht schämen sollte. \\

Oft setzen hierbei Versuchungen an unseren Grundbedürfnissen an. Aber auch unser Handeln und alltägliche
Herausforderungen können zu einer Versuchung führen oder eine Versuchung selbst darstellen. Hierbei kann die Versuchung
durch so ziemlich alle Einflüsse auftreten, die in der realen Welt existieren. Bei Jesus hat der Teufel alle Arsenale
gezogen, von den Bedürfnissen zum Glauben bis hin zu der Macht, die man ergreifen könnte. Daher sollte das Thema nicht
ignoriert werden, aber es ist sinnlos die Versuchung dadurch zu lösen, in dem man die verschiedenen, potentielle
Versuchungen beschreibt, sodass man sie meidet, da diese quasi endlos und abhängig von der eigenen Persönlichkeit sind.
Richtig dagegen ist ein Lösungsorientierer Ansatz, welcher nachfolgend beschrieben wird. \\

Für die Glaubenspraxis können aus der Versuchung Jesu und den vorausgegangenen Betrachtungen Hilfsmittel gezogen
werden:
\begin{itemize}
    \item Gott hilft einen durch Versuchungen auf verschiedene Weisen. Hierbei agiert Gott aber nicht als Flaschengeist, man ist
          von ihn in diesen Bezug abhängig und man darf diese Haltung nicht überschreiten
    \item Ein Gottesvertrauen sowie ein Vertrauen in den eigenen Glauben kann helfen
    \item Die Grundbedürfnisse des Menschen bieten einen Einfallstor für Versuchungen für das Böse. Wenn man sich um diese
          kümmert, können viele Versuchungen präventiv vermieden werden oder gelöst werden
    \item Man darf sich Hilfe nehmen, sei es aus der Gemeinde für spirituelle Hilfe, andere Menschen für intellektuelle Hilfe
          oder Hilfsangebote, die einen helfen die Grundbedürfnisse wieder zu erfüllen und die daraus resultierenden Versuchungen
          wieder zu bekämpfen
    \item Hilfen für Versuchungen und Versuchungen selbst können uns auf den ersten Blick gut erscheinen, aber schaden uns in der
          Wahrheit. Daher sollte man auch immer selbst über die notwendigen Dingen ausführlich und ordentlich reflektieren. Dabei
          sollte man sich aber auch im klaren sein, dass man in solchen Situationen nicht immer zur einwandfreien
          Selbstreflektion fähig ist, da man dort meist besonders unsicher, emotional und verletzlich ist
    \item Versuchungen können uns stärken. Mit so einen positiven Mindset kann sich der Umgang mit Versuchungen v.a.\ auf
          emotionaler Seite einfache gestalten
    \item Am Ende gibt es aber kein garantiertes und universelles Vorgehen oder Allzweckheilmittel für Versuchungen. Es ist und
          wird immer Teil unseres (Glaubens-) Lebens bleiben
\end{itemize}